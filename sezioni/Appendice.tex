\section{Realt\`a Aumentata}
La realtà aumentata consiste nell'arricchimento della percezione sensoriale umana mediante informazioni, in genere manipolate e convogliate elettronicamente, che non sarebbero percepibili con i cinque sensi.
Gli elementi che "aumentano" la realtà possono essere aggiunti attraverso un dispositivo mobile, come uno \textit{smartphone}, con l'uso di un PC dotato di webcam o altri sensori, con dispositivi di visione (per es. occhiali a proiezione sulla retina), di ascolto (auricolari) e di manipolazione (guanti) che aggiungono informazioni multimediali alla realtà già normalmente percepita.

\begin{figure}[H]
	\centering
	\includegraphics[width=1\textwidth]{\docsImg es3.jpg}
	\caption{Esempio di realt\`a aumentata}
	\label{fig:Esempio di realta' aumentata presente nell'app Experenti - Motore}
\end{figure}

Le informazioni "aggiuntive" possono, per\`o, consistere anche in una diminuzione della quantità di informazioni normalmente percepibili per via sensoriale, sempre al fine di presentare una situazione più chiara o più utile o più divertente. Anche in questo caso si parla di AR.

Nella realtà virtuale (\textit{virtual reality}, VR), le informazioni aggiunte o sottratte elettronicamente sono preponderanti, al punto che le persone si trovano immerse in una situazione nella quale le percezioni naturali di molti dei cinque sensi non sembrano neppure essere più presenti e sono sostituite da altre. Nella realtà aumentata (AR), invece, la persona continua a vivere la comune realtà fisica, ma usufruisce di informazioni aggiuntive o manipolate della realtà stessa.

Le informazioni circa il mondo reale che circonda l'utente possono diventare interattive e manipolabili digitalmente.
Le informazioni che “aumentano” la realtà possono essere presenti nella memoria del dispositivo utilizzato, oppure possono essere ricavate da internet in tempo reale.

Prima di essere impiegata in ambito mobile, con applicazioni per \textit{smartphone} e \textit{tablet} o visori da indossare, la realtà aumentata è stata introdotta in ambiti specifici come quello della ricerca, della medicina o nel settore militare. Basti pensare, ad esempio, agli \textit{head-up display} (HUD) equipaggiati sugli aerei da combattimento, che mostrano al pilota informazioni come la distanza dall’obiettivo o l’inclinazione del velivolo, permettendogli di mantenere lo sguardo fisso su ciò che ha di fronte.
In tempi recenti una delle prime app mobile a sfruttare questo approccio è stata Layar. Si tratta di un software che, sfruttando le informazioni di geolocalizzazione fornite dal modulo GPS del dispositivo, e accoppiandole con l’orientamento dello schermo individuato da accelerometro o giroscopio, permette all’utente di inquadrare attraverso la fotocamera l’ambiente circostante, visualizzando icone relative ai punti di interesse presenti nelle vicinanze, esattamente nella direzione in cui si trovano. Questo può risultare utile quando si cerca un ristorante, per capire che strada percorrere per raggiungerlo, oppure in modo da sapere in tempo reale la posizione di altre persone nei dintorni.

La realtà aumentata è una tecnologia applicabile a molti contesti diversi (contrariamente alla realt\`a virtuale che trova le sue principali applicazioni in ambito gaming e multimediale). I principali campi in cui pu\`o essere implementata spaziano dall'\textit{advertising} al \textit{gaming}, dall'edilizia all'arte e all'istruzione.