\setcounter{secnumdepth}{0}
\section{Glossario}

\allsectionsfont{\raggedleft\huge}


\subsection{A}{
	\normalsize
	\begin{longtable}{p{0.25\textwidth} p{0.75\textwidth}} 
		\phantomsection
		\\
		\textbf{Asset}:		&	 E' una risorsa utilizzata in Unity. Un asset puo' essere, per esempio, un modello 3D, uno sprite, una texture, o qualsiasi oggetto di gioco (game object).\\
		\phantomsection
		\\
		\textbf{Avatar}:		&	 E' un modello 3D o un'immagine scelta per rappresentare un'entita' in ambientazioni virtuali.\\
		\phantomsection
	\end{longtable}



\subsection{B}{
	\normalsize
	\begin{longtable}{p{0.25\textwidth} p{0.75\textwidth}} 
		\phantomsection
		\\
		\textbf{Business}:		&	 Insieme delle attività che contribuiscono maggiormente alla produzione del fatturato.\\
		\phantomsection
	\end{longtable}

\subsection{C}{
	\normalsize
	\begin{longtable}{p{0.25\textwidth} p{0.75\textwidth}} 
		\phantomsection

		\\
		\textbf{Client-server}:		&	  La presenza di un server permette ad un certo numero di client di condividerne le risorse, lasciando che sia il server a gestire gli accessi alle risorse per evitare conflitti di utilizzazione.\\
		
		\\
		\textbf{Collider}:		&	 Componente di Unity che serve a gestire le collisioni dell'oggetto con gli altri oggetti di gioco. Collider è la classe base da cui ereditano tutti i vari tipi di collider, ognuno dei quali ha una forma diversa.\\
		
		\\
		\textbf{Cross-Platform}:		&	 Si riferisce ad un linguaggio di programmazione, ad un'applicazione software o ad un dispositivo hardware che funziona su più di un sistema o appunto, piattaforma.\\
		\phantomsection
	\end{longtable}
	
	\subsection{G}{
		\normalsize
		\begin{longtable}{p{0.25\textwidth} p{0.75\textwidth}} 
			\phantomsection
			\\
			\textbf{Gantt}:		&	 Un diagramma di Gantt permette la rappresentazione grafica di un calendario di attività, utile al fine di pianificare, coordinare e tracciare specifiche attività in un progetto dando una chiara illustrazione dello stato d'avanzamento del progetto rappresentato.\\
			\\
			\textbf{GUI}:		&	 E' un tipo di interfaccia utente che consente all'utente di interagire con il device controllando oggetti grafici convenzionali.\\
			\phantomsection
		\end{longtable}
		
	\subsection{I}{
		\normalsize
		\begin{longtable}{p{0.25\textwidth} p{0.75\textwidth}} 
			\phantomsection
			\\
			\textbf{IDE}:		&	 Sta per Integrated Development Enviroment ed indica un ambiente di sviluppo integrato, ovvero un software utilizzato dai programmatori per sviluppare il codice sorgente di un programma.\\
			\phantomsection
		\end{longtable}

	\subsection{P}{
		\normalsize
		\begin{longtable}{p{0.25\textwidth} p{0.75\textwidth}} 
			\phantomsection
			\\
			\textbf{PSD}:		&	 E' un formato di file nativo di Adobe Photoshop per il salvataggio di immagini con le differenti caratteristiche gestite dal programma. \\
			\phantomsection
		\end{longtable}

	\subsection{R}{
		\normalsize
		\begin{longtable}{p{0.25\textwidth} p{0.75\textwidth}} 
			\phantomsection
			\\
			\textbf{Revenue}:		&	 introiti che l'azienda riceve dalla sua normale attivita' di business\\
			\phantomsection
		\end{longtable}
	
	
	\subsection{S}{
		\normalsize
		\begin{longtable}{p{0.25\textwidth} p{0.75\textwidth}} 
			\phantomsection
			\\
			\textbf{Software}:		&	 E' un termine generico che definisce programmi e procedure utilizzati per far eseguire al computer un determinato compito. \\
			\phantomsection
		\end{longtable}
		
	\subsection{T}{
		\normalsize
		\begin{longtable}{p{0.25\textwidth} p{0.75\textwidth}} 
			\phantomsection
			\\
			\textbf{Texture}:		&	 Immagine di qualsiasi tipo utilizzata per rivestire la superficie di un oggetto virtuale, tridimensionale o bidimensionale, con un apposito programma di grafica. \\
			\phantomsection
		\end{longtable}