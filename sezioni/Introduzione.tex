\section{L'azienda}
\subsection{Presentazione}
EXPERENTI srl nasce dalla brillante idea di sfruttare la tecnologia avanzata della realta' aumentata integrandola in un'ottica di business e marketing esperenziale.
La startup e' nata nel 2012 da una collaborazione tra l’Università di Padova e Mentis, società di consulenza strategica. Dopo due anni, e' riuscita a crescere ottenendo degli investimenti e diventando, nel 2014, una vera e propria realta' aziendale ottenendo importanti investimenti che ne hanno permesso una rapida crescita ed espansione al punto da riuscire ad aprire una filiale a New York.

\subsection{Organizzazione aziendale}

\subsection{Prodotti e soluzioni offerte}

\subsection{Processi aziendali}
\subsubsection{Modello di ciclo di vita software}
Come modello di ciclo di vita software, l'azienda ha deciso di adottare una metodologia AGILE. Si riferisce a un insieme di metodi di sviluppo del software emersi a partire dai primi anni 2000 e fondati su insieme di principi comuni, direttamente o indirettamente derivati dai princìpi del "Manifesto per lo sviluppo agile del software". Experenti ha deciso di scegliere questo modello perche', lavorando in un ambiente altamente innovativo, necessita di prediligere le iterazioni con gli individui esterni e la collaborazione con il cliente oltre a una rapida reazione al cambiamento. Questo tipo di modello si prefigge di ottenere software funzionante tralasciando aspetti importanti ma non essenziali quali, per esempio, una documentazione completa.
I principi su cui si basa una metodologia agile che segua i punti indicati dall'Agile Manifesto, sono quattro:
\begin{itemize}
	\item le persone e le interazioni sono più importanti dei processi e degli strumenti (ossia le relazioni e la comunicazione tra gli attori di un progetto software sono la miglior risorsa del progetto);
	\item  più importante avere software funzionante che documentazione (bisogna rilasciare nuove versioni del software ad intervalli frequenti, e bisogna mantenere il codice semplice e avanzato tecnicamente, riducendo la documentazione al minimo indispensabile);
	\item bisogna collaborare con i clienti oltre che rispettare il contratto (la collaborazione diretta offre risultati migliori dei rapporti contrattuali);
	\item bisogna essere pronti a rispondere ai cambiamenti oltre che aderire alla pianificazione (quindi il team di sviluppo dovrebbe essere pronto, in ogni momento, a modificare le priorità di lavoro nel rispetto dell'obiettivo finale).
\end{itemize}

La gran parte dei metodi agili tenta di ridurre il rischio di fallimento sviluppando il software in finestre di tempo limitate chiamate iterazioni che, in genere, durano qualche settimana. Ogni iterazione è un piccolo progetto a sé stante e deve contenere tutto ciò che è necessario per rilasciare un piccolo incremento nelle funzionalità del software: pianificazione, analisi dei requisiti, progettazione, implementazione, test e documentazione.

Il software, all'interno dell'azienda viene sviluppato in finestre di tempo limitate chiamate iterazioni che, in genere, durano dalle 2 alle 4 settimane. Ogni iterazione puo' essere considerata come un piccolo progetto a sé stante e deve contenere tutto ciò che è necessario per rilasciare un piccolo incremento nelle funzionalità del software: pianificazione (planning), analisi dei requisiti, progettazione, implementazione, test e documentazione. La comunicazione con il cliente avviene quotidianamente, fornendo da parte dell'azienda screenshot o video sulle funzionalita' e ottenendo dal cliente feedback e nuove richieste.

Anche se il risultato di ogni singola iterazione non ha sufficienti funzionalità da essere considerato completo deve essere rilasciato e, nel susseguirsi delle iterazioni, deve avvicinarsi sempre di più alle richieste del cliente. Alla fine di ogni iterazione il team rivaluta le priorita' di progetto, viene eseguita una nuova pianificazione e una nuova progettazione in modo da ottenere un sostanziale incremento alla prossima iterazione, fino al completo soddisfacimento del cliente. Se in corso di progettazione in seguito a una richiesta di modifica dei requisiti ci si accorge che alcune funzionalita' richiedono un numero troppo elevato di risorse rispetto a quanto preventivato ci si accorda con il cliente per trovare un compromesso in modo tale che non resti deluso.

I metodi agili preferiscono la comunicazione in tempo reale, preferibilmente faccia a faccia, a quella scritta (documentazione). Il team agile è composto da tutte le persone necessarie per terminare il progetto software. Come minimo il team deve includere i programmatori ed i loro clienti (con clienti si intendono le persone che definiscono come il prodotto dovrà essere fatto: possono essere dei product manager, dei business analysts, o veramente dei clienti).

\subsubsection{Strumenti a supporto dei processi}
\paragraph{Enterprise Resource Planning}

\subsection{Tecnologie utilizzate}

\subsection{Propensione all'innovazione}