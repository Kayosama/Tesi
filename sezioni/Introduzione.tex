\section{L'azienda}
\subsection{Presentazione}
EXPERENTI srl nasce dalla brillante idea di sfruttare la tecnologia avanzata della realt\`a aumentata integrandola in un'ottica di business e marketing esperenziale.
La startup \`e nata nel 2012 da una collaborazione tra l’Università di Padova e Mentis, società di consulenza strategica. Dopo due anni, \`e riuscita a crescere diventando, nel 2014, una vera e propria realt\`a aziendale ottenendo importanti investimenti che ne hanno permesso un rapido sviluppo ed una veloce espansione al punto da riuscire ad aprire una filiale a New York.

\begin{figure}[H]
	\centering
	\includegraphics[width=0.5\textwidth]{\docsImg Experenti-Logo.png}
	\caption{Logo di Experenti srl}
	\label{fig:Logo di Experenti srl}
\end{figure}

\subsection{Organizzazione aziendale}
La filiale produttiva dell’azienda ha sede presso Massanzago (PD) in Via de Faveri 16, e conta nel suo organico due diversi team: 
\begin{itemize}
\item un team con competenze relative al contesto commerciale
\item un team focalizzato sulla parte tecnica e di gestione di progetto.
\end{itemize}

Il primo team si occupa degli aspetti commerciali, di marketing e di immagine dell'azienda. L'obbiettivo principale del team \`e la ricerca e l'aggancio di nuovi clienti, nonch\`e la fidelizzazione dei clienti gi\`a acquisiti. Per fare questo sono presenti diverse figure:
\begin{itemize}
	\item CMO (Chief Marketing Officer): una figura con una preparazione in comunicazione e marketing, che conosce molto bene gli aspetti psicologici. Questa persona si occupa di capire come l'azienda viene percepita dall'esterno e cerca di costruirne un'immagine solida che ispiri fiducia e sicurezza nei possibili clienti. Un ulteriore compito \`e quello di fornire all'azienda la possibilit\`a di partecipare ad eventi e fiere in ambito tecnologico e innovativo e di pubblicizzare l'azienda stessa.
	\item CEO (Chief Executive Officer): \`e l'amministratore delegato dell'azienda. Ne definisce le scelte strategiche, dal modello di business all'approccio al mercato. Segue direttamente le relazioni con i clienti chiave che generano oltre 150000 euro di revenues. Sviluppa le relazioni con investitori e partner.
\end{itemize}

Il secondo team si occupa della parte produttiva e della parte di ricerca e sviluppo. L'obiettivo primario del team \`e quello di soddisfare il cliente realizzando l'applicazione che pi\`u si avvicina alle sue aspettative e nel pi\`u breve tempo possibile, in modo efficace ed efficiente. Obiettivi secondari ma non di minore importanza sono la ricerca di nuove tecnologie e l'implementazione di metodologie che aumentino l'efficacia e l'efficienza con cui viene portato a termine il lavoro.
Le principali figure della squadra sono:
\begin{itemize}
	\item PM (Project Manager): il suo obiettivo essenziale è quello di raggiungere gli obiettivi di progetto, assicurando il rispetto dei costi, dei tempi e della qualità concordati e soprattutto il raggiungimento della soddisfazione del committente. Ha una forte preparazione economica.
	\item CTO (Chief Technology Officer): il suo ruolo \`e quello di monitorare le nuove tecnologie e valutarne il loro potenziale applicato ai prodotti e servizi; ma anche quello di supervisionare i progetti di ricerca per assicurare che portino valore aggiunto alla societ\`a.
	\item sviluppatore software:  si prende cura di pi\`u aspetti del ciclo di vita del software, partendo dall'analisi, passando poi per progettazione e codifica e terminando con il testing e la validazione dell'applicazione. 
\end{itemize} 

Al momento l'azienda si compone di una decina di persone, ma \`e previsto un ampliamento dell'organico per fare fronte al crescente numero di progetti entranti.

\subsection{Prodotti e soluzioni offerte}
\subsubsection{Lo scenario contemporaneo}
Si pu\`o pensare che ormai \`e gi\`a stato tutto scoperto e inventato, ma non \`e cosi perch\`e ogni rivoluzione tecnologica apre un gigantesco universo di possibili applicazioni. 
\\
Quello in cui ci troviamo \`e un secolo che vede la pi\`u rapida espansione ed evoluzione dal punto di vista tecnologico nella storia dell'umanit\`a. La crescita tecnologica si sviluppa ad una velocit\`a tale che quello che oggi esce come una novit\`a tra due mesi viene considerato vecchio e superato.
\\
Inoltre, il modello capitalista adottato dai paesi pi\`u industrializzati e la globalizzazione impongono una continua ricerca ed aggiornamento dei propri prodotti e servizi per risultare pi\`u efficaci ed efficienti sul mercato. Per ottenere ci\`o, soprattutto negli ultimi decenni si sono cercate nuove vie di comunicazione per raggiungere il maggior numero di possibili clienti, investendo moltissimi soldi in campagne pubblicitarie.
\\
Ci troviamo in un periodo temporale in cui siamo assuefatti da pubblicit\`a di ogni genere, al punto da riconoscere un brand esclusivamente osservandone il packaging, oppure osservando il design di un prodotto.
Ma la pubblicit\`a tradizionale sta perdendo efficacia, soprattutto per il fatto che molti mezzi di comunicazione stanno sempre pi\`u cadendo in disuso. Questo \`e, per esempio il caso di riviste e televisori, che si sono viste superare da quello che web e dispositivi mobile stanno sempre pi\`u offrendo.
\\
Lo scopo dell' AR (Augmented Reality) \`e quello di offrire nuove strade di comunicazione, da integrare ai dispositivi mobile (smartphone, tablet e visori).

\subsubsection{Il trend}
\begin{figure}[H]
	\centering
	\includegraphics[width=1\textwidth]{\docsImg gartner.jpg}
	\caption{Grafico Hype Cycle sui principali trend (2014)}
	\label{fig: Grafico Hype Cycle sui principali trend (2014)}
\end{figure}

Quello che emerge da uno studio di Gartner Inc., multinazionale leader mondiale nel campo dell’Information Technology, è una interessante previsione sulla diffusione di alcune tecnologie emergenti. Lo strumento usato da Gartner è lo Hyper Cycle, che consiste in una rappresentazione grafica che mette in scena il ciclo di vita di una tecnologia, dal suo concepimento, alla maturità, alla sua diffusione. Questo, inoltre, viene spesso utilizzato come punto di riferimento nel marketing e nel reporting technology, essendo allegati al grafico i rischi e le opportunità di queste tecnologie.
\\
Nello Hype Cycle possiamo per esempio notare che per la realt\`a aumentata i tempi per un’adozione di massa sono stimati in un periodo compreso tra i 5 ed i 10 anni, e che quindi un investimento nel settore pu\`o portare a una crescita esponenziale dei ricavi dell'azienda. 

\subsubsection{Il core business}
L'obiettivo che vuole raggiungere Experenti \`e quello di sfruttare la tecnologia della realt\`a aumentata per creare un servizio pubblicitario che faccia leva sulle emozioni dei consumatori e su quello che viene definito wow factor (espressione inglese che si riferisce a una qualit\`a o una caratteristica che sorprende; letteralmente elemento sorprendente, o fattore sorpresa). L'obiettivo \`e quello di instaurare nell'utente finale un ricordo piacevole e associare quel ricordo a un marchio o a un prodotto.

\subsubsection{Il prodotto}
Il prodotto vero e proprio che viene creato dall'azienda e fornito ai committenti \`e una applicazione mobile che permette la visualizzazione e l'interazione con alcuni contenuti multimediali, come ad esempio video o modelli 3D, applicati a un "tag" riconosciuto tramite l'ausilio della fotocamera integrata nel device.
Il funzionamento dell'applicazione prevede che, inquadrando con la fotocamera del proprio dispositivo mobile una particolare immagine o un oggetto 3D chiamati Tag, sia possibile visualizzare i contenuti multimediali associati, "aumentando" cosi le informazioni percepite attraverso nuovi canali informativi.
I prodotti sviluppati da Experenti coprono la richiesta di diversi settori, tra i principali:

\begin{itemize}
	\item architettura e arredamento;
	\item comunicazione ed editoria;
	\item turismo e cultura;
	\item istruzione e formazione;
	\item smartcity ed eventi.
\end{itemize}

Sono settori tendenzialmente distanti dal mondo tecnologico, per cui \`e di importanza cruciale un altissimo livello di usabilit\`a dei prodotti. Per cui il prodotto vero e proprio viene creato sulla base delle esperienze precedenti in termini di usabilti\`a e di prestazioni (primo su tutti il risparmio energetico della batteria dei device), e sulla base di chi sar\`a l'utente finale dell'applicazione.

\begin{figure}[H]
	\centering
	\includegraphics[width=1\textwidth]{\docsImg es1.jpg}
	\caption{Esempio di realta' aumentata - Planisfero}
	\label{fig:Esempio di realta' aumentata presente nell'app Experenti - Planisfero}
\end{figure}

\begin{figure}[H]
	\centering
	\includegraphics[width=1\textwidth]{\docsImg es2.jpg}
	\caption{Esempio di realta' aumentata - Motore}
	\label{fig:Esempio di realta' aumentata presente nell'app Experenti - Motore}
\end{figure}


\subsubsection{Tipologie di prodotto}
L'azienda prevede la scelta fra tre diversi tipi di pacchetto, ognuna studiata per venire incontro alle diverse disponibilit\`a di budget dei clienti:

\begin{itemize}
	\item \textbf{inserimento di contenuti interni all'app Experenti}: la prima soluzione e quella pi\`u economica. Questa offerta consiste nell'inserimento di nuovi tag e nuovi contenuti associati all'interno di un visore di realt\`a aumentata gi\`a esistente avente come brand Experenti. Non \`e prevista la modifica della struttura dell'applicazione che funge da contenitore di elementi provenienti da diverse fonti e destinati a diversi utenti. L'applicazione \`e distribuita sia su dispositivi Android che su dispositivi iOS.
	
	\item \textbf{inserimento di contenuti interni ad un'app personalizzata}: \`e la soluzione intermedia, che prevede la creazione di un'app personalizzata sulla base della struttura che possiede l'app di Experenti. E' quindi possibile cambiare nome, logo, palette di colori e contenuti andando a completare le fondamenta standard dell'app.
	
	\item \textbf{inserimento di contenuti interni ad un'app dedicata}: \`e la soluzione meno economica ma pi\`u completa tra quelle presentate. Consiste nella realizzazione di un'app fornendo completa libert\`a di personalizzazione e costruita su misura del cliente.
\end{itemize}

I prodotti tipicamente realizzati sono di due tipologie diverse:
\begin{itemize}
	\item \textbf{app visore}: \`e l'idea su cui si basa l'app Experenti e cio\`e quella di fornire un semplice visualizzatore di contenuti, sia video che modelli 3D. E' data la possibilit\`a di interazione con i contenuti e vengono fornite le funzionalit\`a per scattare screenshot e abilitare il flash della fotocamera.
	
	\item \textbf{app configuratore}: \`e un particolare tipo di applicazione che vede la sua migliore implementazione in ambito di architettura e arredo. Questo particolare tipo di app permette all'utente di "sfogliare" un catalogo di prodotti e di visualizzarli a grandezza naturale per avere una visione d'insieme all'interno del proprio locale o all'esterno. E' possibile vedere i prodotti nelle proprie varianti e di confrontarli tra di loro. L'applicazione prevede tipicamente un men\`u inferiore per scorrere gli elementi e un pannello laterale per visualizzare le informazioni relative a ogni singolo oggetto. E' fornita la possibilit\`a di effettuare ricerche all'interno del catalogo tramite keyword.
\end{itemize}

E' prevista l'introduzione di una nuova tipologia di prodotto ancora in fase di sviluppo che \`e il \textbf{visore di video configurabile da web}. Tutte le altre applicazioni che non rientrano in queste tipologie di prodotto non sono ancora state standardizzate e rientrano nella tipologia di \textbf{applicazioni custom}. 

\begin{figure}[H]
	\centering
	\includegraphics[width=1\textwidth]{\docsImg conf1.jpg}
	\caption{Esempio di app configuratore}
	\label{fig:Esempio di app configuratore}
\end{figure}



\subsection{Processi aziendali}
\subsubsection{Modello di ciclo di vita software}
Come modello di ciclo di vita software, l'azienda ha deciso di adottare una metodologia AGILE. Si riferisce a un insieme di metodi di sviluppo del software emersi a partire dai primi anni 2000 e fondati su insieme di principi comuni, direttamente o indirettamente derivati dai princìpi del "Manifesto per lo sviluppo agile del software". Experenti ha scelto questo modello perch\`e, lavorando in un ambiente altamente innovativo, necessita di prediligere le iterazioni con gli individui esterni e la collaborazione con il cliente oltre a una rapida reazione al cambiamento. Questo tipo di modello si prefigge di ottenere software funzionante tralasciando aspetti importanti ma non essenziali quali, per esempio, una documentazione completa.
I principi su cui si basa una metodologia agile che segua i punti indicati dall'Agile Manifesto, sono quattro:
\begin{itemize}
	\item le persone e le interazioni sono più importanti dei processi e degli strumenti (ossia le relazioni e la comunicazione tra gli attori di un progetto software sono la miglior risorsa del progetto);
	\item  più importante avere software funzionante che documentazione (bisogna rilasciare nuove versioni del software ad intervalli frequenti, e bisogna mantenere il codice semplice e avanzato tecnicamente, riducendo la documentazione al minimo indispensabile);
	\item bisogna collaborare con i clienti oltre che rispettare il contratto (la collaborazione diretta offre risultati migliori dei rapporti contrattuali);
	\item bisogna essere pronti a rispondere ai cambiamenti oltre che aderire alla pianificazione (quindi il team di sviluppo dovrebbe essere pronto, in ogni momento, a modificare le priorità di lavoro nel rispetto dell'obiettivo finale).
\end{itemize}

La gran parte dei metodi agili tenta di ridurre il rischio di fallimento sviluppando il software in finestre di tempo limitate chiamate iterazioni che, in genere, durano qualche settimana. Ogni iterazione è un piccolo progetto a sé stante e deve contenere tutto ciò che è necessario per rilasciare un piccolo incremento nelle funzionalità del software: pianificazione, analisi dei requisiti, progettazione, implementazione, test e documentazione.
\\

\begin{figure}[H]
	\centering
	\includegraphics[width=1\textwidth]{\docsImg Agile.jpg}
	\caption{Flusso dei processi interni secondo la metodologia Agile}
	\label{fig:Flusso dei processi interni}
\end{figure}

Il software, all'interno dell'azienda viene sviluppato in finestre di tempo limitate chiamate iterazioni che, in genere, durano dalle 2 alle 4 settimane. Ogni iterazione pu\`o essere considerata come un piccolo progetto a sé stante e deve contenere tutto ciò che è necessario per rilasciare un piccolo incremento nelle funzionalità del software: pianificazione (planning), analisi dei requisiti, progettazione, implementazione, test e documentazione. La comunicazione con il cliente avviene quotidianamente, fornendo da parte dell'azienda screenshot o video sulle funzionalit\`a e ottenendo dal cliente feedback e nuove richieste.
\\
Anche se il risultato di ogni singola iterazione non ha sufficienti funzionalità da essere considerato completo deve essere rilasciato e, nel susseguirsi delle iterazioni, deve avvicinarsi sempre di più alle richieste del cliente. Alla fine di ogni iterazione il team rivaluta le priorit\`a di progetto, viene eseguita una nuova pianificazione e una nuova progettazione in modo da ottenere un sostanziale incremento alla prossima iterazione, fino al completo soddisfacimento del cliente. Se in corso di progettazione in seguito a una richiesta di modifica dei requisiti ci si accorge che alcune funzionalit\`a richiedono un numero troppo elevato di risorse rispetto a quanto preventivato ci si accorda con il cliente per trovare un compromesso in modo tale che non resti deluso.
\\
I metodi agili preferiscono la comunicazione in tempo reale, preferibilmente faccia a faccia, a quella scritta (documentazione). Il team agile è composto da tutte le persone necessarie per terminare il progetto software. Come minimo il team deve includere i programmatori ed i loro clienti (con clienti si intendono le persone che definiscono come il prodotto dovrà essere fatto: possono essere dei product manager, dei business analysts, o veramente dei clienti).

\subsubsection{Strumenti a supporto dei processi}
\paragraph{Sistemi operativi}
Il lavoro viene svolto in ambiente Microsoft Windows 8.1, anche se a fine agosto \`e iniziato l'aggiornamento di alcune macchine a Windows 10. Una parte del team di sviluppo, invece, lavora in ambiente MacOS principalmente per la compilazione e pubblicazione di app per iOS che ne rendono l'utilizzo necessario. I programmi utilizzati quali Photoshop, Gimp, Unity, sono cross-platform quindi non \`e un problema lo sviluppo su sistemi diversi. 
Per quanto riguarda il server aziendale vediamo la presenza di un sistema Linux cosi come per gli ambienti cloud, in quanto si presta molto bene a quello scopo.
Per il testing delle applicazioni vengono utilizzati dispositivi Android aggiornati all'ultima versione 5.1.1 e dispositivi iOs aggiornati alla versione 8.4.1.

\paragraph{Gestione del versionamento}
Per la gestione del versionamento viene utilizzato internamente Tortoise SVN che \`e un client grafico Subversion. Si \`e scelto il suo utilizzo in quanto, oltre ad essere open source, \`e stato scritto per girare come estensione di Microsoft Windows e quindi perfettamente integrabile nel sistema operativo usato per lo sviluppo software.
I progetti realizzati sono contenuti in un repository che risiede nei server interni e gestito dal reparto tecnico dell'azienda.

\paragraph{Enterprise Resource Planning}
Come sistema di gestione per integrare tutti i processi di business rilevanti, l'azienda ha scelto di utilizzare \textbf{Odoo}, ossia un software ERP OpenSource maturo per la gestione di piccole e medie imprese. 
Odoo integra, tramite moduli, tutti i processi necessari all'impresa come:
\begin{itemize}
	\item gestione della contabilità;
	\item gestione delle risorse umane;
	\item gestione di vendite e acquisti;
	\item gestione dei progetti;
	\item gestione documentale;
\end{itemize}

Odoo è noto per essere molto completo ed estremamente modulare, con più di 1000 moduli disponibili. È basato su una robusta architettura Model-View-Controller, con un server distribuito, workflow flessibili, una GUI dinamica e report personalizzabili.
\\
Le funzionalit\`a principali per cui \`e stato scelto Odoo sono:

\begin{itemize}
	\item \textbf{Kanban Board}: utilizzata per organizzare in modo ottimale il lavoro e avere una visione generale sullo stato dei singoli progetti. E' molto usata soprattutto per il fatto che ci si trova ad agire seguendo un modello di sviluppo molto dinamico e soggetto a continui cambiamenti. L'utilizzo della Kanban Board porta ad eliminare una classe di problemi e sprechi nell'attivit\`a produttiva attraverso un approccio sistematico ovvero creando un ambiente di lavoro che rende difficile commettere errori.
	
	\begin{figure}[H]
		\centering
		\includegraphics[width=0.8\textwidth]{\docsImg Kanban.png}
		\caption{Esempio di Kanban Board}
		\label{fig:Esempio di Kanban Board}
	\end{figure}
	
	\item \textbf{Gestione presenze e richiesta di permessi}: entrate e uscite sono gestite da Odoo cosi come la richiesta di permessi, in questo modo risulta semplice capire la disponibilit\`a di personale a breve e lungo termine.
	
	\item \textbf{Calendarie note condivise}: grazie ai calendari e alle note condivise \`e possibile avere una visione di insieme altrimenti difficile da osservare.	

\end{itemize}

L'utilizzo di Odoo si \`e rivelato di importanza fondamentale per gestire in modo efficiente il tempo del personale, ma anche per avere sempre sotto mano le priorit\`a su un progetto piuttosto che un altro, oppure avere sempre un elenco descrittivo delel attivit\`a da svolgere durante la giornata.
\\
L'utilizzo delle note condivise \`e usato oprattutto in ambito bug fixing, in quanto viene tenuta traccia della soluzione a un particolare bug riscontrato in una applicazione e quindi rintracciabile in futuro da altri che incontrano le stesse problematiche.

\subsection{Tecnologie utilizzate}
\subsubsection{Unity 3D}
Unity 3D \`e una potente e flessibile piattaforma di sviluppo utilizzata per la creazione di giochi 2D e 3D, oltre che per esperienze interattive. E' un ecosistema completo per chiunque miri a costruire un business sulla creazione di contenuti di alto livello. I punti forti che hanno portato alla scelta di questa piattaforma sono essenzialmente:
\begin{itemize}
	\item \textbf{Migliore motore di gioco mobile}: la caratteristica principale \`e che Unity \`e il migliore motore di gioco per lo sviluppo di applicazioni mobile. Permette migliaia di ottimizzazioni per ridurre il peso degli assets e opzioni dedicate per ottenere la migliore resa grafica sui vari dispositivi mobile.
	\item \textbf{Supporto al multiplatform}: Unity che permette la creazione dei contenuti una volta sola e la distribuzione su tutte le principali piattaforme mobile, desktop e console, semplicemente con un click.
	\item \textbf{Partnership di successo}: Unity trae molti benefici dalla forte e positiva partnership instaurata con colossi di piattaforme e costruttori di hardware come Microsoft, Sony, Qualcomm, Intel, Samsung, Oculus VR e Nintento.
\end{itemize}  

\begin{figure}[H]
	\centering
	\includegraphics[width=1\textwidth]{\docsImg unityEditor.jpg}
	\caption{Interfaccia di Unity 3D}
	\label{fig: Interfaccia di Unity 3D}
\end{figure}

\subsubsection{Vuforia SDK}
Vuforia è la piattaforma software che consente le migliori e più creative esperienze di realtà aumentata attraverso gli ambienti del mondo reale, dando alle applicazioni mobile il potere di "vedere" attraverso la fotocamera del device.
La piattaforma Vuforia utilizza un efficiente e stabile riconoscimento delle immagini basato sulla visione artificiale. La visione non \`e intesa solo come acquisizione di una fotografia bidimensionale di un'area ma soprattutto come l'interpretazione del contenuto di quell'area.
E' stato scelto l'utilizzo di Vuforia per il suo pieno supporto ad iOS, Android e Unity 3D.
\\

\begin{figure}[H]
	\centering
	\includegraphics[width=1\textwidth]{\docsImg Vuforia.png}
	\caption{Architettura di Vuforia SDK e flusso di funzionamento delle operazioni di tracciamento delle immagini}
	\label{fig:Architettura di Vuforia SDK e flusso di funzionamento delle operazioni di tracciamento delle immagini
	}
\end{figure}

Di seguito vengono spiegati i componenti dell'architettura di Vuforia SDK AR.

\begin{itemize}
	\item \textbf{Application}: viene eseguita su device. In base ai dati di input avviene l'aggiornamento di stato di alcuni oggetti forniti da Vuforia che servono ad aggiornare l'App Logic e a renderizzare la grafica sullo schermo.
	\item  \textbf{Camera}: assicura che tutti i frame della camera vengano passati al Pixel Format Conversion per ulteriori elaborazioni.
	\item \textbf{Pixel format Conversion}: converte i frame provenienti dal modulo Camera in modo da essere riconosciuti dal rendering e dal tracking dell'OpenGL ES. Questo modulo \`e necessario in quanto le fotocamere sono diverse da dispositivo a dispositivo e forniscono diversi formati di frame.
	\item \textbf{Target Detection}: Vuforia SDK fornisce il rigonoscimento del tag in tre forme:
	\begin{itemize}
		\item tag definiti dall'utente;
		\item tag interni all'applicazione;
		\item tag gestiti in cloud.
	\end{itemize} 
	I tag definiti dall'utente sono definiti usando un'algoritmo interno disponibile in Vuforia SDK. Per la seconda tipologia di tag, \`e necessario il caricamento delle immagini sul portale di sviluppo di Vuforia, per poi scaricare un file da utilizzare in Unity. L'ultima tipologia usa il riconoscimento ricergando i tag tra quelli caricati sul portale di sviluppo di Vuforia.
	\item \textbf{Tracker}: \`e il cuore di Vuforia SDK dove sono scritti tutti gli algoritmi di visione computerizzata per ogni tipologia di tag (immagini, cilindri, etc.). Il modulo si occupa di creare oggetti di stato, in base ai dati ricevuti, che verrano poi utilizzati dall'applicazione sviluppata.
\end{itemize}



\subsection{Propensione all'innovazione}
Come gi\`a accennato in precedenza, Experenti sta cavalcando l'onda di un trend molto caldo, e l'arrivo sul mercato dei visori di realt\`a virtuale e realt\`a aumentata sono un passo in avanti che l'azienda \`e pronta ad affrontare. Durante il mio periodo di stage ho potuto osservare come il team di sviluppo sia sempre aggiornato sulle nuove tecnologie e sulle release di nuove versioni di tecnologie gi\`a utilizzate.
L'azienda ha gi\`a avviato ricerche e sperimentazioni per:
\begin{itemize}
	\item l'utilizzo oggetti tridimensionali reali come tag;
	\item la creazione di applicazioni in realt\`a aumentata senza l'utilizzo di alcun tag;
	\item l'implementazione della realt\`a aumentata su visori AR/VR e quindi la creazione di applicazioni di realt\`a aumentata mista realt\`a virtuale.
\end{itemize}
L'azienda ha gi\`a portato avanti alcuni esempi di realt\`a aumentata sui Google Cardboard con discreti successi.
Anche se l'innovazione \`e un aspetto molto importante per l'azienda, non pu\`o essere prioritaria per il fatto che le risorse sia umane che finanziarie sono ancora limitate e assegnate in primo luogo alla produzione.

\begin{figure}[H]
	\centering
	\includegraphics[width=0.6\textwidth]{\docsImg ARVR.png}
	\caption{Esempio sviluppato su Google Cardboard di AR/VR}
	\label{fig:Esempio sviluppato su Google Cardboard di AR/VR}
\end{figure}