\section{Progetto}
\subsection{Introduzione}
Il progetto ha previsto lo svolgimento di un totale di 320 ore di attività presso l’azienda ospitante, che sono state suddivise in circa 40 ore settimanali. Tali ore si sono svolte internamente all’orario d’ufficio, il quale va dal lunedì a venerdì dalle 9:00 alle 13:00 e dalle 14:30 alle 18:30.
Le date concordate di inizio e fine stage sono state, rispettivamente, 2015-07-13 e 2015-09-11. Durante l’intero periodo di stage, sono stato supervisionato dal tutor interno, incaricato di aiutarmi in caso difficoltà, e dallo staff aziendale, che mi ha fornito supporto in aspetti quali la formazione, permettendomi poi di affiancare il team di sviluppo interno contribuendo attivamente alle attività di progettazione e realizzazione con le competenze acquisite.

\subsection{Descrizione generale}
Lo stage e' stato suddiviso in due parti: la prima parte, prettamente formativa, ha occupato circa il 60\% del periodo di stage, mentre la seconda, che ha occupato il successivo tempo restante, si e' concentrata sulla parte produttiva dell’attività aziendale, in particolar modo sulla parte orientata alla realizzazione di progetti destinati ai clienti esterni. Come tale, l’attività di formazione e' stata opportunamente orientata all’apprendimento, da parte mia, delle meccaniche e delle norme vigenti internamente per lo sviluppo di tali progetti, oltre che alla normale parte di formazione tecnica prevista per portare a termine in maniera opportuna le attività dei progetti stessi.
L’obiettivo finale dello stage e' stato quindi quello di inserirmi come parte integrante del team di sviluppo per i progetti esterni, attribuendomi responsabilità e compiti adeguati al mio ruolo e orientati alle attività di produzione, testing e delivery di app mobile di Realtà Aumentata; la valutazione finale da parte del tutor aziendale e' stata quindi effettuata sulla base sia della qualità sia della quantità delle attività portate a termine nella propria fase produttiva finale, oltre che alla capacità di lavorare correttamente in squadra con l’obiettivo comune di consegnare un prodotto finale nei tempi e nelle modalità stabilite.

\subsection{Dettaglio delle attivita'}
Di seguito vengono elencate in dettaglio le attivita' svolte durante il periodo di stage svolto presso l'azienda ospitante Experenti. Un approfondimento per le principali 

\begin{enumerate}
	\item	Formazione sulle tecnologie utilizzate internamente per lo sviluppo, quali framework e SDK. In particolare:

	\begin{enumerate}	
		\item	Ambiente di sviluppo (IDE) utilizzato (Unity3D) e fondamenti dei sistemi operativi mobile (Android e iOS); 
		\item	Formazione sulle librerie utilizzate internamente per l’elaborazione delle immagini per la realtà aumentata e per il successivo riconoscimento delle stesse in ambiente mobile; 
		\item	Formazione sull’app Experenti: nascita del progetto, funzionamento attuale, obiettivi di sviluppo. Formazione sulle procedure standard applicate internamente.	
	\end{enumerate}

	\item	Realizzazione di un esempio di contenuto in Realtà Aumentata a tema libero. Questo contenuto, il cui sviluppo è stato necessario alla comprensione del flusso di lavoro interno e all'individuazione di determinate problematiche relative all’ambito AR mobile, ha particolari caratteristiche, quali animazioni e/o movimenti di parti specifiche, un certo grado di interattività e prevede parti semplici di grafica GUI (su schermo, in modalità HUD). E' stata richiesta, inoltre, l’individuazione di un tag adatto al riconoscimento dalle fotocamere mobile, possibilmente legato alla tematica che e' stata sviluppata.
	\item	Analisi di casi di studio e app varie già realizzate internamente. Focus particolare sui progetti base già realizzati e sulla loro struttura: progetto base demo, progetto base visore AR, progetto base configuratore. In questa fase e' avvenuta la formazione sul flusso di lavoro standard interno all’azienda e sul normale iter di un progetto commissionato da un cliente, dalla ricezione dei materiali fino alla fase di distribuzione (sia essa una distribuzione ad hoc o una distribuzione pubblica tramite Store mobile) ed e' iniziato l'affiancamento al Project Manager nelle fasi di accettazione materiali. 
	\item	Realizzazione di un’app demo completa. Per app demo si intende un’app a distribuzione solitamente ad hoc (non pubblicata sugli Store) resa disponibile dall’azienda per i propri clienti o reseller, comprendente un numero solitamente limitato di contenuti semplici (3D o video) fruibili dall’utente in realtà aumentata attraverso l’uso di un tag fornito dal cliente stesso. L’app possiede, inoltre, una GUI minimale ma personalizzata con il logo del cliente stesso, nonchè un’icona e una splashscreen anch’esse personalizzate allo stesso modo. Richiesto l'affiancamento al Project Manager fin dalla fase iniziale di ricezione materiali, e prosecuzione poi in autonomia nella fase di sviluppo fino alla fase di rilascio e consegna (previa verifica del risultato prodotto da parte del Tutor Aziendale). L’entità dell’app demo e' stata stabilita dal Project Manager aziendale alcuni giorni prima dell’inizio di questa fase e si e' data preferenza, alla produzione di una demo per un cliente esterno. 
	\item	Inserimento effettivo nel team di sviluppo per i progetti esterni. In questa fase, inizia l'affiancamento al team di sviluppo per i progetti commissionati dai clienti esterni; e' iniziato quindi il coordinamento dal Project Manager aziendale nell’assegnazione di task appositi comprendenti le fasi di sviluppo e testing di intere app semplici o parti di app complesse; si e' preferito assegnare la realizzazione di almeno un’app semplice nella sua interezza commissionata da un cliente esterno. L’assegnazione delle attività e' stato effettuato attraverso il sistema di ticketing utilizzato internamente all’azienda, attraverso il quale e' stato anche richiesto di rendicontare le proprie attività in termini di tempo utilizzato per ciascuna di esse, mentre l’assegnazione dei singoli task e' stata effettuato dal Project Manager aziendale in collaborazione con il tutor aziendale. E' stato valutato positivamente in questa fase la capacità di attenersi alle tempistiche date e il livello di dettaglio fornito nella successiva rendicontazione delle ore, oltre ovviamente alla qualità intrinseca del risultato prodotto. 
\end{enumerate}

