\section{Resoconto dello stage}
\subsection{Pianificazione di progetto}
\subsubsection{Descrizione generale}
Come gia' detto in precedenza, lo stage e' stato suddiviso in due parti: la prima orientata alla formazione su strumenti e tecnologie e la seconda parte orientata alla realizzazione di progetti destinati ai clienti esterni. Come tale, l’attività di formazione e' stata opportunamente orientata all’apprendimento, da parte mia, delle meccaniche e delle norme vigenti internamente per lo sviluppo di tali progetti, oltre che alla normale parte di formazione tecnica prevista per portare a termine in maniera opportuna le attività dei progetti stessi.
L’obiettivo finale dello stage e' stato quindi quello di inserirmi come parte integrante del team di sviluppo per i progetti esterni, attribuendomi responsabilità e compiti adeguati al mio ruolo e orientati alle attività di produzione, testing e delivery di app mobile di Realtà Aumentata; la valutazione finale da parte del tutor aziendale e' stata quindi effettuata sulla base sia della qualità sia della quantità delle attività portate a termine nella fase produttiva finale, oltre che alla capacità di lavorare correttamente in squadra con l’obiettivo comune di consegnare un prodotto finale nei tempi e nelle modalità stabilite.
\\\\
Nel periodo antecedente l'inizio dello stage, insieme al tutor aziendale, sono state concordate le attivita' principali che avrei dovuto svolgere durante il periodo seguente della durata di 2 mesi. Nella descrizione delle attivita', riportata nella sezione successiva, e' stata fornita una descrizione molto generica per quanto riguarda il progetto principale che avrei dovuto seguire in quanto non era ancora chiaro a priori se ci sarebbe stata o meno la possibilita' di seguire un progetto commissionato dall'esterno.
\\\\
La dislocazione temporale delle attivita e' stata rappresentata graficamente in un Diagramma di Gantt che mi ha aiutato ad avere sempre una visione accurata sullo stato del mio stage, in particolare su eventuali ritardi. Rispetto al diagramma concordato nel piano di lavoro, il mio stage e' partito dopo 2 giorni rispetto a quanto concordato a causa di un'indisponibilita' del tutor aziendale, per cui e' stata rifatta la pianificazione tenendo conto di questo ritardo.

\begin{figure}[H]
	\centering
	\includegraphics[width=1\textwidth]{\docsImg Gantt.png}
	\caption{Diagramma di Gantt delle attivita'}
	\label{fig:Diagramma di Gantt delle attivita'}
\end{figure}

Il Diagramma di Gantt riportato in figura \ref{fig:Diagramma di Gantt delle attivita'} mostra piuttosto fedelmente quanto svolto durante il periodo in azienda ed eventuali anticipi sulla tabella di marcia sono stati riempiti con approfondimenti sulle tecnologie e sessioni di ricerca e sviluppo su visori Google Cardboard. Non ci sono stati, invece, ritardi su quanto preventivato.

\subsubsection{Dettaglio delle attivita'}
Di seguito vengono elencate in dettaglio le attivita' svolte durante il periodo di stage svolto presso l'azienda ospitante Experenti. Un approfondimento per le principali. 

\begin{enumerate}
	\item	Formazione sulle tecnologie utilizzate internamente per lo sviluppo, quali framework e SDK. In particolare:
	
	\begin{enumerate}	
		\item	Ambiente di sviluppo (IDE) utilizzato (Unity3D) e fondamenti dei sistemi operativi mobile (Android e iOS); 
		\item	Formazione sulle librerie utilizzate internamente per l’elaborazione delle immagini per la realtà aumentata e per il successivo riconoscimento delle stesse in ambiente mobile; 
		\item	Formazione sull’app Experenti: nascita del progetto, funzionamento attuale, obiettivi di sviluppo. Formazione sulle procedure standard applicate internamente.	
	\end{enumerate}
	
	\item	Realizzazione di un esempio di contenuto in Realtà Aumentata a tema libero. Questo contenuto, il cui sviluppo è stato necessario alla comprensione del flusso di lavoro interno e all'individuazione di determinate problematiche relative all’ambito AR mobile, ha particolari caratteristiche, quali animazioni e/o movimenti di parti specifiche, un certo grado di interattività e prevede parti semplici di grafica GUI (su schermo, in modalità HUD). E' stata richiesta, inoltre, l’individuazione di un tag adatto al riconoscimento dalle fotocamere mobile, possibilmente legato alla tematica che e' stata sviluppata.
	\item	Analisi di casi di studio e app varie già realizzate internamente. Focus particolare sui progetti base già realizzati e sulla loro struttura: progetto base demo, progetto base visore AR, progetto base configuratore. In questa fase e' avvenuta la formazione sul flusso di lavoro standard interno all’azienda e sul normale iter di un progetto commissionato da un cliente, dalla ricezione dei materiali fino alla fase di distribuzione (sia essa una distribuzione ad hoc o una distribuzione pubblica tramite Store mobile) ed e' iniziato l'affiancamento al Project Manager nelle fasi di accettazione materiali. 
	\item	Realizzazione di un’app demo completa. Per app demo si intende un’app a distribuzione solitamente ad hoc (non pubblicata sugli Store) resa disponibile dall’azienda per i propri clienti o reseller, comprendente un numero solitamente limitato di contenuti semplici (3D o video) fruibili dall’utente in realtà aumentata attraverso l’uso di un tag fornito dal cliente stesso. L’app possiede, inoltre, una GUI minimale ma personalizzata con il logo del cliente stesso, nonchè un’icona e una splashscreen anch’esse personalizzate allo stesso modo. Richiesto l'affiancamento al Project Manager fin dalla fase iniziale di ricezione materiali, e prosecuzione poi in autonomia nella fase di sviluppo fino alla fase di rilascio e consegna (previa verifica del risultato prodotto da parte del Tutor Aziendale). L’entità dell’app demo e' stata stabilita dal Project Manager aziendale alcuni giorni prima dell’inizio di questa fase e si e' data preferenza, alla produzione di una demo per un cliente esterno. 
	\item	Inserimento effettivo nel team di sviluppo per i progetti esterni. In questa fase, inizia l'affiancamento al team di sviluppo per i progetti commissionati dai clienti esterni; e' iniziato quindi il coordinamento dal Project Manager aziendale nell’assegnazione di task appositi comprendenti le fasi di sviluppo e testing di intere app semplici o parti di app complesse; si e' preferito assegnare la realizzazione di almeno un’app semplice nella sua interezza commissionata da un cliente esterno. L’assegnazione delle attività e' stato effettuato attraverso il sistema di ticketing utilizzato internamente all’azienda, attraverso il quale e' stato anche richiesto di rendicontare le proprie attività in termini di tempo utilizzato per ciascuna di esse, mentre l’assegnazione dei singoli task e' stata effettuato dal Project Manager aziendale in collaborazione con il tutor aziendale. E' stato valutato positivamente in questa fase la capacità di attenersi alle tempistiche date e il livello di dettaglio fornito nella successiva rendicontazione delle ore, oltre ovviamente alla qualità intrinseca del risultato prodotto. 
\end{enumerate}

\begin{center}
	
	\begin{longtable}{c| p{0.7\textwidth}| c}

		\textbf{Sezione} & \textbf{Descrizione} & \textbf{Ore di lavoro}\\ \cline{1-3}
		\phantomsection
		1.1&  Formazione su ambienti di sviluppo&  40 \\
				\phantomsection
		1.2&  Formazione su librerie utilizzate&  28 \\
				\phantomsection
		1.3&  Formazione sull’app Experenti&  12 \\
				\phantomsection
		2&  Realizzazione di un contenuto di realtà aumentata a tema libero&   56\\	
				\phantomsection
		3&  Analisi su progetti già realizzati internamente e formazione su flusso di lavoro interno&   40\\
				\phantomsection	
		4&  Realizzazione di un’app demo completa&   24\\
				\phantomsection					
		5&  Inserimento nel team di sviluppo e realizzazione di un’app nella sua interezza&   120\\ \cline{1-3}
				\phantomsection
		  & \textbf{TOTALE} & \textbf{320}  \\\\
		  		\caption{Tabella relativa alle ore dedicate per ciascuna attivita'}\\
	\end{longtable}
		
\end{center}

\subsection{Studio delle tecnologie e strumenti}
In questa sezione, vengono spiegate le attivita' di apprendimento svolte per imparare l'utilizzo delle nuove tecnologie e degli strumenti usati.
\subsubsection{Unity 3D}
Unity, come gia' detto in precedenza e' un sistema cross-platform per lo sviluppo di giochi composto da un game engine e da un IDE integrato. Unity viene usato internamente all'azienda per lo sviluppo di app mobile distribuite su Android e iOS.
\\
Unity nel suo sito fornisce un grosso supporto agli sviluppatori fornendo una documentazione completa e una sezione ben fornita di tutorial testuali e video suddivisi per categoria.\\
Inizialmente, ho dovuto seguire una parte di video tutorial riguardanti l'interfaccia di Unity, lo scripting, la gestione della fisica, animazioni e gestione della GUI. Questa primo periodo si e' svolto integrando, oltre alla visione, anche la prova diretta sull'editor in modo da assimilare meglio i concetti appresi.\\
Nel caso in cui volessi approfondire un argomento oppure non lo ritenessi abbastanza chiaro, avevo sempre la possibilita' di ottenere una spiegazione da parte del tutor aziendale, il quale si e' dimostrato sempre molto disponibile anche nel ripetere piu' volte lo stesso concetto.\\
In questa parte di formazione, dopo aver seguito e implementato un tutorial riguardante l'animazione di un avatar, di mia iniziativa, ho effettuato il porting dell'applicazione su Android gestendo touch e multi-touch sullo schermo e impostando i movimenti dell'avatar basandoli sull'acelerometro del dispositivo mobile.

\begin{figure}[H]
	\centering
	\includegraphics[width=1\textwidth]{\docsImg tutorial.jpg}
	\caption{Implementazione in Unity del tutorial riguardante l'animazione di un avatar 3D di cui successivamente e' stato eseguito il porting su Android}
	\label{fig:tutorial riguardante l'animazione di un avatar 3D}
\end{figure}
 
\subsubsection{Vuforia SDK}
Terminata la parte di formazione su Unity 3D, e' iniziata la parte di preparazione relativa a Vuforia SDK, l'SDK utilizzato dall'azienda per l'implementazione della realta' aumentata. Nello specifico, il team tecnico si e' occupato di spiegarmi come funziona l'SDK, e come funziona nello specifico il plugin di Unity, grazie alla quale è possibile operare all’interno di un unico ambiente di lavoro. Le attivita' principali svolte in questo lasso di tempo sono state l'implementazione di modelli 3D e di video associati ai tag, e lo studio sul riconoscimento e la creazione di tag ottimali.

\subsubsection{Photon Unity Networking}
Photon Unity Networking (PUN) e' un framework di Unity per l'implementazione del multiplayer realtime neigiochi o nelle applicazioni sviluppate. Le applicazioni svilupapte con Photon vengono eseguite su un server cloud proprietario. Quindi, le operazioni di scaling e di service hosting sonogestite interamente da PUN, permettendo allo sviluppatore di concentrarsi puramente sulla costruzione dell'applicazione. Tutti i prodotti Photon Cloud sono basati su un'architettura client-to-server, che e' la soluzione ottimale per il gaming onlire rispetto a una connessione peer-to-peer.\\
Photon e' un package scaricabile dall'Asset Store, e nella sua versione gratuita prevede l'accesso concorrente fino a 20 utenti sulla stessa stanza. \\\\
Di seguito vengono riportati esempi di codice per mostrare la semplicita' di utilizzo del framework.\\
\begin{itemize}
	\item \textbf{Connessione al server}: La connessione al server si basa sul passaggio di una stringa contenente la versione dell'applciazione. Puo' essere usata per dividere gruppi di client.
\begin{lstlisting}
PhotonNetwork.ConnectUsingSettings("1.0");
\end{lstlisting}

	\item \textbf{Accesso a una stanza}: Per prendere parte a una partita esistente basta la seguente riga di codice specificando il nome della stanza in cui si vuole entrare.
\begin{lstlisting}
PhotonNetwork.JoinRoom("RoomName");
\end{lstlisting}

	\item \textbf{Creare una stanza}: per creare una stanza basta fornire il nome, dare la possibilita' o meno di essere trovata da altri utenti, fornire la possibilita' agli altri di entrare, e il numero massimo di giocatori.
\begin{lstlisting}
public void OnConnectedToMaster()
{
	PhotonNetwork.CreateRoom("RoomName", true, true, 4);
}

\end{lstlisting}	
	
\end{itemize}

Lo studio di Photon Unity Networking non mi e' stato imposto dall'azienda, ma e' stato un approfondimento che ho voluto fare di mia iniziativa per lo sviluppo dell'esempio di contenuto in realta' aumentata a tema libero di cui parlero' successivamente.

\begin{figure}[H]
	\centering
	\includegraphics[width=0.9\textwidth]{\docsImg Photon.png}
	\caption{Architettura generale del framework Photon Unity Networking}
	\label{fig:Architettura generale del framework Photon Unity Networking}
\end{figure}

PUN si e' dimostrato uno strumento molto potente e relativamente di facile utilizzo. Creare un semplice scambio di dati tra diversi client e' risultato piuttosto semplice. Il livello di difficolta' e' salito quando ho cercato di aumentare il numero di informazioni passate e il numero di oggetti da "osservare". Essendo uno studio non richiesto dall'azienda, ho preferito non spendere troppo tempo in approfondimenti ma piuttosto avere un'idea chiara del funzionamento di base.

\subsubsection{Esempio di contenuto in realta' aumentata a tema libero}

\subsection{Svolgimento delle attivita'}
\subsubsection{Analisi dei requisiti}
\subsubsection{Progettazione}
\subsubsection{Implementazione}
\subsubsection{Verifica e validazione}

\subsection{Livello di completezza raggiunto}