\section{Strategia Aziendale}
\subsection{Motivazione dello stage}
Lo stage ha potuto svolgersi grazie all'evento STAGE-IT 2015 che ha permesso l'incontro tra le imprese e gli studenti che sarebbero entrati a breve in stage nel mondo del lavoro con specifico riferimento al settore ICT. L'evento ha favorito un'occasione di conoscenza reciproca mediante colloqui individuali.
\\
Experenti sta vivendo un momento di forte crescita, e ha visto nell'ultimo periodo un aumento del numero di progetti in ingresso. Per fare fronte alla richiesta, l'azienda ha deciso di espandere il suo organico anche in una possibile ottica di inserimento post-stage.
\\
Il team di Experenti richiedeva un laureando in Informatica che possedesse un’ottima capacità di programmazione ad oggetti, la conoscenza di C\# e una propensione per la parte di progettazione propedeutica al coding vero e proprio.
Inoltre, era apprezzata una qualche esperienza con modellazione, rendering 3D e con il motore grafico Unity 3D.
\\
Per una azienda avviata da soli due anni, e' di fondamentale importanza gestire in modo ottimale le risorse, soprattutto quelle finanziarie. Per cui, l'azienda ha valutato positivamente il fatto di poter prendere uno stagiaire a tempo limitato senza obbligo di retribuzione, in modo da avere a disposizione ulteriori forze nell'immediato per gestire il notevole numero di progetti entranti in quel periodo. 
\\
Non e' stata una scelta dettata esclusivamente dalla necessita' di manodopera, pero', in quanto il tempo di formazione dello stagiaire comportava un dispendio iniziale di risorse, in quanto era necessario l'affidamento di un tutor aziendale per l'insegnamento delle metodologie, dell'utilizzo degli strumenti e delle best practises presenti in azienda.
\subsection{Obiettivo dello stage}
Lo stage prevedeva la suddivisione delle attivita' in due parti: la prima prettamente formativa, ha occupato circa il 60\% del periodo di stage, mentre la seconda, che ha occupato il successivo tempo restante, si e' concentrata sulla parte produttiva dell’attività aziendale, in particolar modo sulla parte orientata alla realizzazione di progetti destinati ai clienti esterni.
\\
Come \textbf{obiettivo minimo} era richiesto di sviluppare almeno un singolo contenuto complesso in realtà aumentata (ovvero: non video AR semplice e non 3D statico AR) da inserire all’interno di un’app commissionata da un cliente esterno. 
\\
Mentre, come \textbf{obiettivo massimo} era richiesto di sviluppare un’intera app visore di AR, completa di tutti i suoi contenuti semplici e complessi e della propria grafica, dalla fase di accettazione dei materiali in entrata fino alla fase di consegna della beta finale al cliente.
\\
Il progetto che avrei dovuto seguire non era stabilito sin da subito, ma e' stato concordato a stage gia' avviato, in seguito all'ingresso di un progetto commissionato da Cora' Divisione Parquet, di cui parlero' in seguito.
\\
Il progetto consisteva nella realizzazione di un configuratore di arredo in realta' aumentata e nella gestione di un avatar 3D che effettuasse una presentazione iniziale e si occupasse di seguire l'utente durante l'utilizzo dell'app con una spiegazione sulle varie categorie di prodotto.
\\\\
Entrando nel dettaglio, era richiesto di partire da un configuratore di prodotto di base, che consiste in un'applicazione tramite la quale gli utenti possono scegliere un modello di prodotto e le caratteristiche desiderate, e una volta definiti possono mandare una e-mail di richiesta preventivo oppure essere rimandati al sito web. 
\\
Il prodotto di base da cui bisognava partire era un configuratore, gia' realizzato, di stufe comprensivo di menu' inferiore per la selezione delle categorie e dei prodotti e un pannello laterale mostrante la descrizione di ogni prodotto. Nel configuratore, inoltre, era gia' presente uno script per gestire l'auto-focus della camera del device e un mirino con un piccolo pulsante per scaricare il tag nel caso non fosse gia' disponibile all'utente.
\\
Definita la base di partenza, la prima parte del progetto, era la creazione della GUI personalizzata partendo da una grafica in formato PSD. Successivamente, bisognava inserire i primi prodotti all'interno del configuratore e quindi gestire i singoli dati riguardanti un prodotto in modo da fornire, in futuro, l'eventuale possibilita' di effettuare ricerche tramite l'inserimento di keyword in un'apposita casella di input testuale.
\\
Si richiedeva, inoltre, l'implementazione di una feature nuova, per fornire la gesture di pinch-to-zoom realizzata ad-hoc per una pavimentazione irregolare che e' quella del parquet. Era richiesta un'estensione della superficie coperta dal parquet in concomitanza ad una particolare gesture (in questo caso un "pizzico" sullo schermo del device) applicata all'oggetto tridimensionale che si stava visualizzando.
\\
\\
La seconda parte del progetto consisteva nella gestione di un avatar umanoide in realta' aumentata e di un modello tridimensionale rappresentante un ventaglio in legno da usare come sipario per la comparsa dell'avatar. L'avatar doveva effettuare una presentazione iniziale dell'applicazione e delle tipologie di prodotto, e si voleva poterlo richiamare successivamente tramite un apposito pulsante per richiedere spiegazioni riguardanti particolari categorie di prodotto concordate con il committente.
\subsection{Vincoli imposti}
\subsubsection{Vincoli tecnologici}
\paragraph{Android}
Android è un sistema operativo personalizzabile per dispositivi mobili sviluppato da Google Inc. basato su kernel Linux.
È stato progettato principalmente per smartphone e tablet, con interfacce utente specializzate per televisori (Android TV), automobili (Android Auto), orologi da polso (Android Wear), occhiali (Google Glass), e altri.
È per la quasi totalità Free and Open Source Software, ed è distribuito sotto i termini della licenza libera Apache 2.0.\\
Android dispone di una vasta comunità di sviluppatori che realizzano applicazioni con l'obiettivo di aumentare le funzionalità dei dispositivi. Queste applicazioni sono scritte soprattutto in linguaggio di programmazione Java.

\paragraph{Unity 3D}

\paragraph{Vuforia SDK}

\paragraph{Monodevelop}

\paragraph{C\#}

\subsubsection{Vincoli metodologici}

\subsubsection{Vincoli temporali}
Il progetto prevedeva che lo stagiaire svolgesse un totale di 320 ore di attività presso l’azienda ospitante, suddivise in circa 40 ore settimanali soggette a possibili variazioni nel caso di scadenze aziendali o di impegni di varia natura da parte dello studente. Tali ore si dovevano svolgere internamente all’orario d’ufficio, dal lunedì a venerdì dalle 9:00 alle 13:00 e dalle 14:30 alle 18:30.
\\
Le prime date concordate di inizio e fine stage sono state, rispettivamente, 2015-07-09 e 2015-09-09. In seguito a problematiche sorte da parte del tutor aziendale a ridosso della data di inizio stage si e' deciso di riconcordare nuovamente le date di inizio e fine stage, rispettivamente, 2015-07-13 e 2015-09-11. 
\\
Il periodo concordato e' stato suddiviso in due sezioni temporali della stessa dimensione:
\begin{itemize}
	\item la prima parte prettamente formativa, consisteva nello studio delle tecnologie utilizzate e nella realizzazione di un piccolo progetto per l'assimilazione dei concetti appresi;
	\item la seconda parte, iniziata in conclusione della parte formativa, consisteva nella realizzazione del progetto vero e proprio.
\end{itemize}

\subsection{Prospettive}
Come descritto in precedenza, Experenti vuole ampliare il suo organico, inserendo nel team alcune nuove figure con una preparazione in ambito informatico e grafico. Il numero sempre crescente di progetti entranti rendono necessario l'inserimento di sviluppatori Android e iOS che seguano attivamente i progetti dalla fase di raccolta dei materiali, alla fase di progettazione, codifica e testing.