\section{Strategia Aziendale}
\subsection{Motivazione dello stage}
Lo stage ha potuto svolgersi grazie all'evento STAGE-IT 2015 che ha permesso l'incontro tra le imprese e gli studenti che sarebbero entrati a breve in stage nel mondo del lavoro con specifico riferimento al settore ICT. L'evento ha favorito un'occasione di conoscenza reciproca mediante colloqui individuali.
\\
Experenti sta vivendo un momento di forte crescita, e ha visto nell'ultimo periodo un aumento del numero di progetti in ingresso. Per fare fronte alla richiesta, l'azienda ha deciso di espandere il suo organico anche in una possibile ottica di inserimento post-stage.
\\
Il team di Experenti richiedeva un laureando in Informatica che possedesse un’ottima capacità di programmazione ad oggetti, la conoscenza di C\# e una propensione per la parte di progettazione propedeutica al coding vero e proprio.
Inoltre, era apprezzata una qualche esperienza con modellazione, rendering 3D e con il motore grafico Unity 3D.
\\
Per una azienda avviata da soli due anni, \`e di fondamentale importanza gestire in modo ottimale le risorse, soprattutto quelle finanziarie. Per cui, l'azienda ha valutato positivamente il fatto di poter prendere uno stagiaire a tempo limitato senza obbligo di retribuzione, in modo da avere a disposizione ulteriori forze nell'immediato per gestire il notevole numero di progetti entranti in quel periodo. 
\\
Non \`e stata una scelta dettata esclusivamente dalla necessit\`a di manodopera, per\`o, in quanto il tempo di formazione dello stagiaire comportava un dispendio iniziale di risorse, in quanto era necessario l'affidamento di un tutor aziendale per l'insegnamento delle metodologie, dell'utilizzo degli strumenti e delle best practises presenti in azienda.
\subsection{Obiettivo dello stage}
Lo stage prevedeva la suddivisione delle attivit\`a in due parti: la prima prettamente formativa, ha occupato circa il 60\% del periodo di stage, mentre la seconda, che ha occupato il successivo tempo restante, si \`e concentrata sulla parte produttiva dell’attività aziendale, in particolar modo sulla parte orientata alla realizzazione di progetti destinati ai clienti esterni.
\\
Come \textbf{obiettivo minimo} era richiesto di sviluppare almeno un singolo contenuto complesso in realtà aumentata (ovvero: non video AR semplice e non 3D statico AR) da inserire all’interno di un’app commissionata da un cliente esterno. 
\\
Mentre, come \textbf{obiettivo massimo} era richiesto di sviluppare un’intera app visore di AR, completa di tutti i suoi contenuti semplici e complessi e della propria grafica, dalla fase di accettazione dei materiali in entrata fino alla fase di consegna della beta finale al cliente.
\\
Il progetto che avrei dovuto seguire non era stabilito sin da subito, ma \`e stato concordato a stage gi\`a avviato, in seguito all'ingresso di un progetto commissionato da Cor\`a Divisione Parquet, di cui parler\`o in seguito.
\\
Il progetto consisteva nella realizzazione di un configuratore di arredo in realt\`a aumentata e nella gestione di un avatar 3D che effettuasse una presentazione iniziale e si occupasse di seguire l'utente durante l'utilizzo dell'app con una spiegazione sulle varie categorie di prodotto.
\\\\
Entrando nel dettaglio, era richiesto di partire da un configuratore di prodotto di base, che consiste in un'applicazione tramite la quale gli utenti possono scegliere un modello di prodotto e le caratteristiche desiderate, e una volta definiti possono mandare una e-mail di richiesta preventivo oppure essere rimandati al sito web. 
\\
Il prodotto di base da cui bisognava partire era un configuratore, gi\`a realizzato, di stufe comprensivo di men\`u inferiore per la selezione delle categorie e dei prodotti e un pannello laterale mostrante la descrizione di ogni prodotto. Nel configuratore, inoltre, era gi\`a presente uno script per gestire l'auto-focus della camera del device e un mirino con un piccolo pulsante per scaricare il tag nel caso non fosse gi\`a disponibile all'utente.
\\
Definita la base di partenza, la prima parte del progetto, era la creazione della GUI personalizzata partendo da una grafica in formato PSD. Successivamente, bisognava inserire i primi prodotti all'interno del configuratore e quindi gestire i singoli dati riguardanti un prodotto in modo da fornire, in futuro, l'eventuale possibilit\`a di effettuare ricerche tramite l'inserimento di keyword in un'apposita casella di input testuale.
\\
Si richiedeva, inoltre, l'implementazione di una feature nuova, per fornire la gesture di pinch-to-zoom realizzata ad-hoc per una pavimentazione irregolare che \`e quella del parquet. Era richiesta un'estensione della superficie coperta dal parquet in concomitanza ad una particolare gesture (in questo caso un "pizzico" sullo schermo del device) applicata all'oggetto tridimensionale che si stava visualizzando.
\\
\\
La seconda parte del progetto consisteva nella gestione di un avatar umanoide in realt\`a aumentata e di un modello tridimensionale rappresentante un ventaglio in legno da usare come sipario per la comparsa dell'avatar. L'avatar doveva effettuare una presentazione iniziale dell'applicazione e delle tipologie di prodotto, e si voleva poterlo richiamare successivamente tramite un apposito pulsante per richiedere spiegazioni riguardanti particolari categorie di prodotto concordate con il committente.

\subsection{Vincoli imposti}
Per tutta la durata di svolgimento dello stage sono state imposte alcune condizioni da rispettare, spiegate di seguito, divise per tipologia.

\subsubsection{Vincoli tecnologici}
\paragraph{Android}
Android è un sistema operativo personalizzabile per dispositivi mobili sviluppato da Google Inc. basato su kernel Linux.
È stato progettato principalmente per smartphone e tablet, con interfacce utente specializzate per televisori (Android TV), automobili (Android Auto), orologi da polso (Android Wear), occhiali (Google Glass), e altri.
È per la quasi totalità Free and Open Source Software, ed è distribuito sotto i termini della licenza libera Apache 2.0.\\
Android dispone di una vasta comunità di sviluppatori che realizzano applicazioni con l'obiettivo di aumentare le funzionalità dei dispositivi. Queste applicazioni sono scritte soprattutto in linguaggio di programmazione Java.
\\

\begin{figure}[H]
	\centering
	\includegraphics[width=1\textwidth]{\docsImg Usage.jpg}
	\caption{Utilizzo dei sistemi operativi mobile nel secondo quarto del 2014 (dato fornito da Net Applications)}
	\label{fig:Utilizzo dei sistemi operativi mobile nel secondo quarto del 2014}
\end{figure}

Si \`e scelto di sviluppare il progetto, in prima istanza, per i dispositivi Android su richiesta del committente (anche se la release dell'app \`e avvenuta anche su iOS) e soprattutto perch\`e, mentre Apple vive in un ambiente chiuso e ben definito, per quanto riguarda Android ci si trova a dover fare i conti con la diversit\`a di hardware dei dispositivi e diverse risoluzioni degli schermi. Per lo sviluppo dell'app si predilige Android, in quanto possono sorgere pi\`u problematiche, e dato che Unity, comunque, offre la possibilit\`a di sviluppare l'app solo una volta e distribuirla a pi\`u dispositivi diversi.


\paragraph{Unity 3D}
Come gi\`a spiegato in precedenza, Unity 3D è un ambiente di sviluppo per la creazione di giochi. Come prima cosa fornisce un potente engine di supporto. Il motore in questione offre il supporto completo per tutti gli aspetti quali rendering grafico, effetti di luce, creazione di terrain, simulazioni fisiche, implementazione dell’audio, funzionalità di rete e, cosa pi\`u importante, un sistema di scripting.
Unity semplifica di molto la vita del programmatore con piccole feature di qualit\`a, come per esempio la "live preview", che consente di vedere dal vivo quello che si sta creando, con un solo click. Inoltre, mette a disposizione un sistema di gestione delle risorse da usare nel progetto efficace ed intuitivo.
I formati di file supportati dall'engine sono molteplici. 
\\Per i modelli 3D si ha il pieno supporto ai file generati da:

\begin{multicols}{3}
\begin{itemize}
	\item Maya;
	\item 3D Studio Max;
	\item Cinema4D;
	\item Blender;
	\item SketchUp;
	\item Cheetah;
	\item Lightwave;
	\item XSI;
	\item Carrara;
	\item Wavefront Obj.
\end{itemize}
\end{multicols}
Mentre per quanto riguarda le immagini, i formati supportati sono:

\begin{multicols}{3}
	\begin{itemize}
	\item JPEG;
	\item PNG;
	\item GIF;
	\item BMP;
	\item TGA;
	\item IFF;
	\item PICT;
\end{itemize}
\end{multicols}

Anche a livello di audio il sistema si difende piuttosto bene: 
\begin{multicols}{2}
	\begin{itemize}
		\item MP3;
		\item Ogg Vorbis;
		\item AIFF;
		\item WAV.
	\end{itemize}
\end{multicols}

Infine, per i video si ha supporto a:
\begin{multicols}{3}
	\begin{itemize}
		\item MP4;
		\item MPG;
		\item AVI;
		\item MOV;
		\item ASF;
		\item MPEG.
	\end{itemize}
\end{multicols}

Uno dei punti di forza di Unity 3D \`e il fatto di essere gratuito in quasi tutte le sue features. Tuttavia,
per particolari scopi o esigenze di pubblicazione, ci sono alcuni strumenti o parti del sistema che sono a pagamento. Unity \`e la versione base che viene rilasciata gratuitamente con quasi tutte le funzionalit\`a pi\`u importanti a disposizione. Unity Pro, la versione apagamento, consente allo sviluppatore di usufruire di diverse features non presenti nella versione normale; su tutte l'assenza dello splash screen di Unity, cio\`e della schermata iniziale mostrante il logo di Unity.
Per usi commerciali, l'azienda necessita l'utilizzo della versione Pro, anche per il fatto che questa versione mette a disposizione utilissimi effetti di render su texture, di post processing e su luci e ombre.
\\

Nella figura \ref{fig:Versione base di Unity} si pu\`o osservare una tipica schermata di lavoro di Unity nella versione base (non Unity Pro) composta da cinque sezioni ben distinte:
\begin{enumerate}
	\item \textbf{Scene}: La scena \`e lo spazio di lavoro in cui vengono posizionati gli oggetti di gioco e in cui \`e possibile avere una visione "grezza" di come apparir\`a la nostra applicazione. In questa zona \`e possibile effettuare azioni sugli oggetti, quali modificare la scala, cambiare la loro posizione sugli assi, ruotarli o modificare le loro ancore e pivot.
	\item \textbf{Game}: questa sezione \`e dove viene mostrata la "live preview" dell'applicazione. Schiacciando il pulsante "play", infatti, sar\`a possibile avviare l'applicazione e interagire con essa, con la possibilit\`a di poter usufruire di tutte le gesture mobile tramite l'uso di tastiera e mouse. Per simulare la camera del dispositivo viene usata una webcam collegata al computer.
	\item \textbf{Hierarchy}: \`e la gerarchia di oggetti di gioco istanziati. Qui si possono tenere sotto controllo tutti gli oggetti presenti nella scena e modificarne le parentele.
	\item \textbf{Project}: nella sezione Project vengono inserite tutte le risorse che si vuole rendere disponibili nel progetto come texture, script, modelli 3D, etc. Qui vengono inseriti anche gli oggetti non istanziati nella scena, ma che devono essere istanziati runtime.
	\item \textbf{Inspector}: l'inspector \`euna sezione importantissima in cui si possono modificare tutti i parametri di ogni oggetto di gioco e agganciare nuove componenti come per esempio animator, collider, etc.
\end{enumerate}

\begin{figure}[H]
	\centering
	\includegraphics[width=1\textwidth]{\docsImg Unity.png}
	\caption{Versione base di Unity}
	\label{fig:Versione base di Unity}
\end{figure}

Anche il layout delle sezioni da utilizzare in Unity mi \`e stato imposto, in quanto pi\`u persone lavorano sullo stesso progetto ed \`e necessario non rimanere disorientati da una diversa disposizione dell'interfaccia.

\paragraph{OpenGL ES}
OpenGL ES è uno standard industriale per la programmazione grafica 3D su dispositivi mobile. Khronos Group, un conglomerato che include marchi come ATI, NVIDIA ed Intel si preoccupa di definire ed estendere lo standard.
Attualmente esistono molte versioni diverse delle specifiche OpenGL ES. La versione 1.0 venne ricalcata sulla versione 1.3 di OpenGL, mentre la versione 1.1 si basa su OpenGL 1.5 e la 2.0 è definita in relazione a OpenGL 2.0. La versione di OpenGL ES utilizzata era la 2.0 poich\`e, ormai, la gran parte dei dispositivi Android supporta quel tipo di libreria grafica.

\paragraph{Vuforia SDK}
Come gi\`a descritto in precedenza, \`e stato imposto il vincolo di utilizzo di Vuforia SDK, rispetto al suo principale competitor Metaio. La scelta \`e stata fatta alle origini dell'azienda, ed \`e stata fatta sulla base del supporto fornito in quanto documentazione, supporto e tutorial. Vuforia \`e una piattaforma completa che offre feature di spessore quali:

\begin{itemize}
	\item Rilevamento istantaneo dei tag locali;
	\item Riconoscimento cloud fino a 1 milione di tag simultanei;
	\item Riconoscimento e tracking di testo stampato;
	\item Tracking contemporaneo fino a 5 tag;
	\item Risultati eccezionali con condizioni ambientali sfavorevoli come tag semi-coperti e carenza di luce.
\end{itemize}

Rispetto a Metaio, Vuforia SDK presenta un tracking migliore e una migliore integrazione con Unity 3D, che ne hanno dettato la scelta rispetto all'utilizzo di Metaio e altri competitor minori. 


\paragraph{C\#}
Per la scrittura del codice \`e stato imposto l'utilizzo di C\#, in quanto \`e un linguaggio molto simile al Java, di cui avevo gi\`a una buona preparazione. L'alternativa all'utilizzo di C\# sarebbe stata Javascript, un linguaggio molto potente ma che avrebbe richiesto ulteriore tempo di formazione. Essendo che pi\`u persone collaborano sullo stesso progetto, si \`e ritenuto indispensabile la piena comprensione da parte di tutti i membri del team di sviluppo del linguaggio utilizzato.

\subsubsection{Vincoli metodologici}
Durante tutta l’attività di stage, \`e stato imposta la gestione del versionamento tramite l'utilizzo di TortoiseSVN, inserendo le componenti in un repository interno al server locale dell'azienda. Non mi \`e stato fornito alcun vincolo su quando effettuare i commit, la cui gestione \`e stata lasciata a me.

\subsubsection{Vincoli temporali}
Il progetto prevedeva che lo stagiaire svolgesse un totale di 320 ore di attività presso l’azienda ospitante, suddivise in circa 40 ore settimanali soggette a possibili variazioni nel caso di scadenze aziendali o di impegni di varia natura da parte dello studente. Tali ore si dovevano svolgere internamente all’orario d’ufficio, dal lunedì a venerdì dalle 9:00 alle 13:00 e dalle 14:30 alle 18:30.
\\
Le prime date concordate di inizio e fine stage sono state, rispettivamente, 2015-07-09 e 2015-09-09. In seguito a problematiche sorte da parte del tutor aziendale a ridosso della data di inizio stage si \`e deciso di riconcordare nuovamente le date di inizio e fine stage, rispettivamente, 2015-07-13 e 2015-09-11. 
\\
Il periodo concordato \`e stato suddiviso in due sezioni temporali della stessa dimensione:
\begin{itemize}
	\item la prima parte prettamente formativa, consisteva nello studio delle tecnologie utilizzate e nella realizzazione di un piccolo progetto per l'assimilazione dei concetti appresi;
	\item la seconda parte, iniziata in conclusione della parte formativa, consisteva nella realizzazione del progetto vero e proprio.
\end{itemize}

\subsection{Prospettive}
Come descritto in precedenza, Experenti vuole ampliare il suo organico, inserendo nel team alcune nuove figure con una preparazione in ambito informatico e grafico. Il numero sempre crescente di progetti entranti rendono necessario l'inserimento di sviluppatori Android e iOS che seguano attivamente i progetti dalla fase di raccolta dei materiali, alla fase di progettazione, codifica e testing.
\\
Il settore altamente innovativo e appena nato della realt\`a aumentata, comporta la possibilit\`a di lavorare su progetti all'avanguardia in ambito AR. Basta osservare il lavoro svolto nel mio stage per accorgersi delle continue migliorie che gli si possono applicare. Experenti presta molta attenzione nella ricerca di clienti che possano offrire motivazione nel cercare soluzioni per il raggiungimento di un grado di innovazione sempre maggiore e cercare, quindi, di essere sempre un passo avanti rispetto ai competitor.