\section{Valutazione Retrospettiva}
Segue ora una valutazione personale sul periodo di stage trascorso.
\subsection{Soddisfacimento obiettivi}
Lo stage si e' svolto nel completo rispetto di tutti i vincoli imposti, concludendosi in data 11/09/2015. \\
Per quanto riguarda gli obiettivi dello stage, essi sono variati durante lo svolgimento in quanto non era garantita la disponibilita' di seguire un progetto commissionato dall'esterno.\\
Come obiettivo massimo è richiesto di sviluppare un’intera app visore di AR, completa di tutti i suoi contenuti semplici e complessi e della propria grafica, dalla fase di accettazione dei materiali in entrata fino alla fase di consegna della beta finale al cliente.\\
Questo obiettivo e' stato pienamente raggiunto, anzi si e' quasi arrivati al release dell'applicazione sugli store, avvenuto pochi giorni dopo il termine del mio stage.\\
Tutti i requisiti dell'applciazione sono stati soddisfatti e i contenuti richiesti implementati al meglio delle mie attuali competenze.\\
L'applicazione realizzata e' piaciuta molto a Cora' Divisione Parquet e soprattutto all'amministratore delegato Ettore Cora', il quale si e' complimentato per l'ottimo lavoro svolto. E' certo che il progetto verra' ampliato con nuovi contenuti e nuovi prodotti continuando a usufruire dei servizi offerti da Experenti.\\
\subsection{Conoscenze acquisite}
E' difficile descrivere la totalita' delle cose che ho imparato nel periodo di due mesi di stage effettuato, in quanto e' stata un'esperienza completamente nuova e totalmente diversa da quanto gia' appreso nei vari progetti didattici di team.\\
A livello tecnologico ho potuto arricchirmi di strumenti e tecnologie innovative e all'avanguardia che sicuramente saranno sempre piu' richieste a livello curricolare con la prossima uscita di device portatili di ultima generazione e visori di AR/VR. In particolare:

\begin{itemize}
	\item \textbf{Unity}:
	\item \textbf{Vuforia}:
	\item \textbf{C\#}:
\end{itemize}

Ma non solo, in quanto la crescita piu' grande e' avvenuta nell'applicazione di processi aziendali e nell'applicazione del lavoro di squadra.
\subsection{Distanza tra universita' e lavoro}
\subsection{Valutazione personale}
\newpage
\subsection{Screenshot finali}

	\begin{figure}[H]
		\centering
		\includegraphics[width=1\textwidth]{\docsImg s5.jpg}
		\caption{Presentazione iniziale - Ventaglio di legno in apertura}
		\label{fig:Presentazione iniziale - Ventaglio di legno in apertura}
	\end{figure}
	
	\begin{figure}[H]
		\centering
		\includegraphics[width=1\textwidth]{\docsImg s2.jpg}
		\caption{Presentazione iniziale - Avatar}
		\label{fig:Presentazione iniziale - Avatar}
	\end{figure}
		
	\begin{figure}[H]
		\centering
		\includegraphics[width=1\textwidth]{\docsImg s6.jpg}
		\caption{Tutorial - Istruzioni}
		\label{fig:Tutorial - Istruzioni}
	\end{figure}	
	
	\begin{figure}[H]
		\centering
		\includegraphics[width=1\textwidth]{\docsImg s1.jpg}
		\caption{Configuratore - Menu' inferiore aperto e Avatar in presentazione}
		\label{fig:Configuratore - Menu' inferiore aperto e Avatar in presentazione}
	\end{figure}
	
	\begin{figure}[H]
		\centering
		\includegraphics[width=1\textwidth]{\docsImg s3.jpg}
		\caption{Configuratore - Pannello info aperto}
		\label{fig:Configuratore - Pannello info aperto}
	\end{figure}
		
	\begin{figure}[H]
		\centering
		\includegraphics[width=1\textwidth]{\docsImg s4.jpg}
		\caption{Configuratore - Menu' dei Credits aperto}
		\label{fig:Configuratore - Menu' dei Credits aperto}
	\end{figure}