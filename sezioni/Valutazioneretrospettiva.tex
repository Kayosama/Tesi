\section{Valutazione Retrospettiva}
Segue ora una valutazione personale sul periodo di stage trascorso.
\subsection{Soddisfacimento obiettivi}
Lo stage si \`e svolto nel completo rispetto di tutti i vincoli imposti, concludendosi in data 11/09/2015. \\
Per quanto riguarda gli obiettivi dello stage, essi sono variati durante lo svolgimento in quanto non era garantita la disponibilit\`a di seguire un progetto commissionato dall'esterno.\\
Come obiettivo massimo era stato richiesto di sviluppare un’intera app visore di AR, completa di tutti i suoi contenuti semplici e complessi e della propria grafica, dalla fase di accettazione dei materiali in entrata fino alla fase di consegna della beta finale al cliente.\\
Questo obiettivo \`e stato pienamente raggiunto, anzi si \`e quasi arrivati al release dell'applicazione sugli store, avvenuto pochi giorni dopo il termine del mio stage.\\
Tutte le funzionalit\`a richieste sono state implementate e l’interfaccia consegnata \`e risultata completamente funzionante, questo ha permesso di ottenere un giudizio positivo da parte del tutor e del committente.\\
\`E stata soddisfatta la totalit\`a dei requisiti obbligatori e la quasi toatilit\`a di quelli desiderabili. Non ho potuto seguire alcune richieste da parte del cliente che sono giunte gli ultimi giorni di stage.\\
Ho rispettato tutti i vincoli, sia tecnologici, sia quelli di metodo che temporali. Va evidenziato, per\`o, che il numero di ore preventivate per ogni attivit\`a risulta diverso da quello stabilito nel piano di progetto.\\
\newpage
\begin{center}
	
	\begin{longtable}{c| p{0.7\textwidth}| c}
		
		\textbf{Sezione} & \textbf{Descrizione} & \textbf{Ore di lavoro}\\ \cline{1-3}
		\phantomsection
		1.1&  Formazione su ambienti di sviluppo&  32 \\
		\phantomsection
		1.2&  Formazione su librerie utilizzate&  20 \\
		\phantomsection
		1.3&  Formazione sull’app Experenti&  8 \\
		\phantomsection
		2&  Realizzazione di un contenuto di realtà aumentata a tema libero&   56\\	
		\phantomsection
		3&  Analisi su progetti già realizzati internamente e formazione su flusso di lavoro interno&   32\\
		\phantomsection	
		4&  Realizzazione di un’app demo completa&   12\\
		\phantomsection					
		5&  Inserimento nel team di sviluppo e realizzazione di un’app nella sua interezza&   160\\ \cline{1-3}
		\phantomsection
		& \textbf{TOTALE} & \textbf{320}  \\\\
		\caption{Tabella retrospettiva relativa alle ore dedicate per ciascuna attivit\`a}\\
	\end{longtable}
\end{center}
Come si evince dalla tabella, le ore effettivamente impiegate per la parte di formazione sono inferiori rispetto a quelle preventivate. Questo perch\`e su alcuni argomenti trattati, quelli cio\`e riguradanti Android, ero gi\`a abbastanza preparato, in quanto \`e stata una tecnologia ampiamente utilizzata durante il progetto didattico di Ingegneria del \textit{Software}.\\
Inoltre, non ho avuto particolari problemi di apprendimento dai \textit{tutorial} seguiti, guadagnando qualche ora per ogni attivit\`a svolta in ambito formativo.\\
Il tempo guadagnato mi ha permesso di dedicare pi\`u tempo alla realizzazione del progetto, fornendo un grado di completezza maggiore rispetto a quanto preventivato. Infatti, dagli obiettivi, si vede che era richiesta una demo, mentre la demo \`e stata consegnata una settimana prima della fine dello stage, permettendomi di lavorare su alcune delle ultime richieste del cliente prima della \textit{release}. \\

\subsection{Conoscenze acquisite}
Le conoscenze acquisite durante lo stage sono di due caratteri: tecnologiche e metodologiche. \\
A livello tecnologico ho potuto arricchirmi di strumenti e tecnologie innovative e all'avanguardia che sicuramente saranno sempre pi\`u richieste a livello curricolare con la prossima uscita di device portatili di ultima generazione e visori di AR/VR. In particolare:

\begin{itemize}
	\item \textbf{Unity}: Strumento che sin da subito ha saputo mostrare il suo grandissimo potenziale nella realizzazione di applicazioni \textit{cross-platform\gloss}. Dall'inizio dello stage sono state utilizzate versioni di Unity dalla 4.6.0 alla 5.1.2, mostrandomi l'evoluzione che questo strumento ha subito per agevolare sempre di pi\`u lo sviluppo di applicazioni.\\
	Le principali nozioni apprese sono state:
	\begin{itemize}
		\item \textbf{Fisica}: Ho imparato a gestire la fisica degli oggetti e la loro interazione con il mondo che li circonda. Ho imparato ad applciare forze agli oggetti in modo da spostarli o lanciarli via.
		
		\item \textbf{\textit{Collider\gloss}}: Imparare a gestire e interagire con i collider \`e stata una delle cose pi\`u utili apprese, in quanto i collider servono per gestire le interazioni degli oggetti tra di loro e con l'esterno.
		
		\item \textbf{\textit{Raycast}}: La gestione dei raycast \`e fondamentale per permettere l'interazione dell'utente con gli oggetti virtuali tramite i tap su schermo.
		
		\item \textbf{Animazioni e \textit{Animator}}: Sono due aspetti di Unity che ho imparato a conoscere in profondit\`a, in quanto largamente utilizzati all'interno del progetto.
		
		\item \textbf{Creazione di GUI\gloss}: creare una buona GUI\gloss\ \`e di importanza fondamentale, non solo internamente a Unity ma per qualsiasi tipo di applicazione. Ho compreso come gestire testi, immagini, pulsanti, pannelli e \textit{slider}.
	\end{itemize}
	
	\item \textbf{Vuforia}: Ci\`o che ho appreso di Vuforia \`e stata principalmente la gestione di un singolo tag, ma anche una base per il \textit{multi-target}. Le principali nozioni apprese sono state:
	\begin{itemize}
		\item \textbf{Gestione Tag}: Ho appreso come gestire il caricamento dei tag sul portale di Vuforia e come riconoscere i tag ottimali e quelli pessimi. Ho imparato a utilizzare i tag internamente a Unity grazie al \textit{plugin} fornito da Vuforia SDK e ad assegnargli dei contenuti.
		
		\item \textbf{Modelli 3D}: I contenuti utilizzati per la realt\`a aumentata sono stati principalmente modelli 3D semplici, ma ho imparato anche a gestire modelli 3D complessi, quali sono gli avatar\gloss\ .
		
		\item \textbf{Video}: Oltre a contenuti 3D ho appreso come collegare i video a un tag e come implementare la funzionalit\`a di video \textit{follow}, che permette a un video in esecuzione di disancorarsi dal tag e di seguire il device quando viene perso il \textit{tracking}.
		
		\item \textbf{\textit{Extended Tracking}}: Ho appreso quando \`e meglio usare questa funzionalit\`a permessa da Vuforia SDK e come viene applicata. Inoltre, ho imparato come ottimizzare al meglio gli oggetti che sfruttano questa opzione del tag.
	\end{itemize}
	
	\item \textbf{C\#}: \`E un linguaggio di programmazione utilizzato per lo scripting delle componenti interne a Unity. Il linguaggio \`e risultato molto simile al Java e al C++, ma con meno simbolismi e meno elementi decorativi ma comunque orientato agli oggetti in modo nativo. Non ci sono stati problemi nell'apprendimento di questo linguaggio, in quanto la mia preparazione in Java e C++ \`e stata pi\`u che sufficiente.
\end{itemize}
\noindent
Ho potuto, inoltre, imparare a utilizzare metodologie e strumenti all'avanguardia, come:

\begin{itemize}
	\item \textbf{Metodologia Agile}: Durante tutto il periodo di stage ho potuto apprendere i meccanismi su cui si basa la metodologia Agile, comprendendone le priorit\`a e le tempistiche di lavoro. Ho potuto, inoltre, capire il grado di esperienza necessario per la migliore attuazione di questa metodologia e la sinergia interna al gruppo richiesta per il raggiungimento degli obiettivi finali.
	
	\item \textbf{\textit{Kanban Board}}: Ho imparato a usare questo strumento per la gestione dei progetti, trovandolo fondamentale in ambito Agile, in quanto permette di quantificare il tempo dedicato ad ogni attivit\`a e permette a tutti i membri del \textit{team} di sviluppo di conoscere lo stato dei progetti e sapere a cosa stanno lavorando i colleghi.
	
	\item \textbf{\textit{Repository} e SVN}: L'uso di \textit{Repository} e del versionamento era una pratica gi\`a conosciuta e utilizzata che ho potuto consolidare nel periodo di stage. Inoltre, ho imparato a usare uno strumento di \textit{subversion} quale Tortoise SVN prima sconosciuto.\\
	Mentre per i progetti didattici svolti il livello di dettaglio nei \textit{commit} era relativamente importante, mi sono trovato a dover fornire un livello considerevolmente maggiore dato che un progetto pu\`o essere ripreso anche dopo mesi dalla sua ultima modifica.
\end{itemize}
\noindent
Infine, la mia crescita personale \`e avvenuta anche grazie a un miglioramento nelle mie capacit\`a in fatto di:

\begin{itemize}
	\item \textbf{Comunicazione}: Sia interna alla squadra di sviluppo, sia con gli altri componenti di Experenti. Ho imparato a comunicare in modo preciso riguardo sia a tematiche tecniche, sia per quanto riguarda la comunicazione esterna con il cliente avvenuta tramite il \textit{Project Manager}.\\
	\`E stato fondamentale imparare a collaborare di squadra, utlizzando una metodologia Agile che impone frequente comunicazione sia interna che esterna.
	
	\item \textbf{Responsabilit\`a}: Durante lo svolgimento del progetto mi \`e stata data grande responsabilit\`a sulla buona riuscita del prodotto. Ho avuto campo libero per agire, richiedere modifiche progettuali sull'usabilit\`a, e per gestire al meglio il mio tempo. Il piano di lavoro riguardante il progetto, infatti, \`e stato stilato insieme al \textit{Project Manager} secondo le tempistiche da me comunicate.
	
	\item \textbf{Quantificazione del lavoro}: Ho potuto apprendere come quantificare al meglio il tempo necessario per un determinato lavoro. Ovviamente, questa abilit\`a \`e stata ottenuta solo verso la fine dello stage al ripetersi di attivit\`a gi\`a svolte in precedenza.
\end{itemize}

\subsection{Distanza tra universit\`a e lavoro}
La distanza tra universit\`a e mondo del lavoro \`e ancora ampia, ma il corso di studi seguito ha permesso di avvicinare queste due realt\`a. In particolare, grazie a:

\begin{itemize}
	\item \textbf{Seminari}: Durante l'ultimo anno del corso di studi ho potuto prendere parte a numerosi seminari, tecnologici e non, che mi hanno permesso di gettare le basi per l'apprendimento delle nuove tecnologie utilizzate e di capire le metodologie di lavoro aziendali. Seminari di questo genere sono serviti per avere una visione della tipologia di figure ricercate da parte delle aziende, e delle conoscenze richieste. Il seminario tenuto dai ragazzi che lavorano a Google mi ha permesso, inoltre, di capire meglio come funzionano i colloqui in un'azienda esigente che lavora con un alto tasso innovativo.
	
	\item \textbf{Corsi mirati}: Sempre durante la mia carriera universitaria ho potuto seguire corsi obbligatori e opzionali, mirati all'avvicinamento con il mondo del lavoro. Questo \`e l'esempio del corso di Ingegneria del \textit{Software} che grazie a un complesso progetto mi ha permesso di avere un assaggio del lavoro in \textit{team} in condizione di risorse limitate con tempi stringenti.\\
	Un altro corso di fondamentale importanza per la mia formazione \`e stato quello di Gestione di Imprese Informatiche, che mi ha mostrato come \`e composta un'azienda e soprattutto come si crea un'azienda. Il corso, inoltre, prevedeva un progetto di simulazione di creazione di \textit{startup} in un team di 4 persone.
	
	\item \textbf{Eventi}: L'universit\`a prepara eventi di incontro tra studenti e aziende come STAGE-IT, un evento che mi ha permesso di vedere numerose realt\`a aziendali e di incontrare Experenti.\\
	Oltre a questo, la mia partecipazione ad altri eventi quali, per esempio SMAU, mi hanno permesso di farmi conoscere dalle varie aziende del settore ICT e a mia volta di capire dove puntano le aziende e che tipo di figure cercano.\\
\end{itemize}
\noindent
D'altra parte, alcuni elementi fanno da barriera per un passaggio agevolato da universit\`a a lavoro. I principali sono:

\begin{itemize}
	\item \textbf{Tecnologie obsolete}: I corsi tenuti in ambito di programmazione vengono tenuti usando linguaggi di programmazione ormai obsoleti e caduti in disuso. Questo \`e l'esempio del linguaggio Perl usato nel corso di Tecnologie Web. Per avvicinare al mondo del lavoro \`e richiesta almeno una base solida sui pi\`u moderni linguaggi e framework quali: Ruby on Rails, Angular.js, Javascript.
	
	\item \textbf{Corsi fortemente teorici}: Un elemento molto importante dovrebbe essere la maggiore presenza di corsi mirati all'integrazione nel mondo lavorativo. Questo avviene in troppi pochi corsi di cui molti opzionali.
\end{itemize}
\noindent
Gli elementi che rendono ancora ampio questo divario non sono esclusivamente colpe dell'Universit\`a ma anche richieste troppo esigenti da parte delle aziende.\\
Durante i numerosi colloqui svolti mi sono trovato in molte situazioni in cui l'azienda richiedeva la conoscenza di decine di tecnologie di cui alcune altamente innovative. La maggior parte delle aziende, quindi, richiede figure gi\`a formate, cosa che l'Universit\`a non pu\`o procurare visto il settore in cui lavoriamo in continua evoluzione.  
\newpage
\subsection{Valutazione personale}
Per quanto riguarda la mia personale esperienza di stage posso ritenermi pienamente soddisfatto. Tramite l'evento STAGE-IT ho avuto modo di conoscere varie realt\`a aziendali, partendo dalle startup formate da 4-5 persone arrivando ad aziende con all'attivo pi\`u di 80 dipendenti. Ho avuto modo di effettuare diversi colloqui e di visitare in sede diverse aziende. Tutte le realt\`a che ho visto mi hanno incantato. Ho potuto visitare l'ambiente sobrio di startup visitate all'interno di centri di ricerca, assaporare uno stile indipendente e altamente innovativo, ma ho potuto vedere gli ampi e ben arredati spazi di una grande azienda, della seriet\`a con cui lavorano e la voglia che hanno di crescere e migliorare continuamente.\\
Fortunatamente, ho potuto osservare quanta speranza \`e ancora riposta in noi giovani, in un settore in cui le persone e soprattutto il valore di una persona viene ancora riconosciuto come una risorsa fondamentale per la crescita di un'azienda.\\
Alla fine, la mia scelta di svolgere lo stage ad Experenti \`e stata ampiamente ripagata dalle numerose soddisfazioni ottenute e dal modo in cui sono stato accolto all'interno dell'azienda, trovando un ambiente accogliente e persone aperte nei miei confronti.\\
Un'esperienza sicuramente appagante che mi ha fatto scoprire le mie vere capacit\`a, rendendole disponibili anche agli altri.
Tutto il team di Experenti si \`e dimostrato professionale e disponibile nel supportarmi nella mia formazione, fornendomi spiegazioni dedicate e rispondendo ad ogni mio dubbio. Sono stati molto propensi alle mie proposte, sia contestualmente al progetto da svolgere, sia per l'integrazione di nuovi strumenti di supporto.\\
Concludo affermando la mia soddisfazione per il corso di studi scelto, visto non solo in ottica di nozioni apprese ma in una visione di maturazione a livello personale e professionale.
\newpage
\subsection{Screenshot finali}

	\begin{figure}[H]
		\centering
		\includegraphics[width=1\textwidth]{\docsImg s5.jpg}
		\caption{Presentazione iniziale - Ventaglio di legno in apertura}
		\label{fig:Presentazione iniziale - Ventaglio di legno in apertura}
	\end{figure}
	
	\begin{figure}[H]
		\centering
		\includegraphics[width=1\textwidth]{\docsImg s2.jpg}
		\caption{Presentazione iniziale - Avatar}
		\label{fig:Presentazione iniziale - Avatar}
	\end{figure}
		
	\begin{figure}[H]
		\centering
		\includegraphics[width=1\textwidth]{\docsImg s6.jpg}
		\caption{Tutorial - Istruzioni}
		\label{fig:Tutorial - Istruzioni}
	\end{figure}	
	
	\begin{figure}[H]
		\centering
		\includegraphics[width=1\textwidth]{\docsImg s1.jpg}
		\caption{Configuratore - Men\`u inferiore aperto e Avatar in presentazione}
		\label{fig:Configuratore - Menu' inferiore aperto e Avatar in presentazione}
	\end{figure}
	
	\begin{figure}[H]
		\centering
		\includegraphics[width=1\textwidth]{\docsImg s3.jpg}
		\caption{Configuratore - Pannello info aperto}
		\label{fig:Configuratore - Pannello info aperto}
	\end{figure}
		
	\begin{figure}[H]
		\centering
		\includegraphics[width=1\textwidth]{\docsImg s4.jpg}
		\caption{Configuratore - Men\`u dei Credits aperto}
		\label{fig:Configuratore - Menu' dei Credits aperto}
	\end{figure}