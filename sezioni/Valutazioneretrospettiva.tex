\section{Valutazione Retrospettiva}
Segue ora una valutazione personale sul periodo di stage trascorso.
\subsection{Soddisfacimento obiettivi}
Lo stage si \`e svolto nel completo rispetto di tutti i vincoli imposti, concludendosi in data 11/09/2015. \\
Per quanto riguarda gli obiettivi dello stage, essi sono variati durante lo svolgimento in quanto non era garantita la disponibilit\`a di seguire un progetto commissionato dall'esterno.\\
Come obiettivo massimo è richiesto di sviluppare un’intera app visore di AR, completa di tutti i suoi contenuti semplici e complessi e della propria grafica, dalla fase di accettazione dei materiali in entrata fino alla fase di consegna della beta finale al cliente.\\
Questo obiettivo \`e stato pienamente raggiunto, anzi si \`e quasi arrivati al release dell'applicazione sugli store, avvenuto pochi giorni dopo il termine del mio stage.\\
Tutti i requisiti dell'applciazione sono stati soddisfatti e i contenuti richiesti implementati al meglio delle mie attuali competenze.\\
L'applicazione realizzata \`e piaciuta molto a Cor\`a Divisione Parquet e soprattutto all'amministratore delegato Ettore Cor\`a, il quale si \`e complimentato per l'ottimo lavoro svolto. E' certo che il progetto verr\`a ampliato con nuovi contenuti e nuovi prodotti continuando a usufruire dei servizi offerti da Experenti.\\
\subsection{Conoscenze acquisite}
E' difficile descrivere la totalit\`a delle cose che ho imparato nel periodo di due mesi di stage effettuato, in quanto \`e stata un'esperienza completamente nuova e totalmente diversa da quanto gi\`a appreso nei vari progetti didattici di team.\\
A livello tecnologico ho potuto arricchirmi di strumenti e tecnologie innovative e all'avanguardia che sicuramente saranno sempre pi\`u richieste a livello curricolare con la prossima uscita di device portatili di ultima generazione e visori di AR/VR. In particolare:

\begin{itemize}
	\item \textbf{Unity}: Strumento che sin da subito ha saputo mostrare il suo grandissimo potenziale nella realizzazione di applicazioni cross-platform. L'apprendimento non \`e stato immediato ma ha richiesto diverso tempo, anche solo per comprendere le dinamiche principali. Alcuni aspetti non sono stati colti subito, ma solo dopo l'applicazione in esempi reali. Comunque, la mia esperienza con altri software simili mi ha permesso di non trovarmi disorientato durante l'apprendimento. Dall'inizio dello stage sono state utilizzate versioni di Unity dalla 4.6.0 alla 5.1.2, facendomi mostrare l'evoluzione che questo strumento ha subito per agevolare sempre di pi\`u lo sviluppo di applicazioni.
	\item \textbf{Vuforia}: Vuforia \`e la tecnologia pi\`u innovativa studiata durante lo stage. Il suo apprendimento e utilizzo \`e stato abbastanza facile grazie alla presenza del plugin di Vuforia per Unity. Allo studio \`e stata data grandissima importanza, in quanto questa tecnologia era di diretto interesse per l'applicazione sviluppata.
	\item \textbf{C\#}: E' un linguaggio di programmazione utilizzato per lo scripting delle componenti interne a Unity. Il linguaggio \`e risultato molto simile al Java e al C++, ma con meno simbolismi e meno elementi decorativi ma comunque orientato agli oggetti in modo nativo. Non ci sono stati problemi nell'apprendimento di questo linguaggio, in quanto la mia preparazione in Java e C++ \`e stata pi\`u che sufficiente.
\end{itemize}

Ma il mio apprendimento non si \`e fermato esclusivamente all'ambito tecnologico, in quanto la crescita pi\`u grande \`e avvenuta nell'applicazione di processi aziendali e nella gestione del lavoro di squadra.\\
Ho potuto apprendere dall'interno come funziona un'azienda appena avviata, delle figure necessarie alla gestione di ogni aspetto, da quello economico a quello di comunicazione, e di come sia lavorare utilizzando una metodologia Agile.

\subsection{Distanza tra universit\`a e lavoro}
Dal mio punto di vista, la distanza che separa universit\`a e mondo del lavoro \`e ancora molto ampia.  Bisogna riconoscere, per\`o, che il corso di laurea da me seguito \`e riuscito a prepararmi molto bene per affrontare al meglio questo salto. Merito soprattutto di alcuni corsi che mi hanno insegnato, pi\`u che linguaggi di programmazione, delle metodologie di programmazione volte alla qualit\`a del prodotto sviluppato e all'esendibilit\`a del codice scritto.\\
Corsi come Ingegneria del Software mi hanno permesso, grazie a progetti didattici, di lavorare in un team di persone con capacit\`a diverse pura avendo lo stesso background di studi. In questi progetti ho potuto svolgere diversi ruoli e vedere la realizzazione di un progetto complesso da pi\`u punti di vista.\\
Da parte del mio corso di studi \`e per\`o necessario un continuo aggiornamento sugli insegnamenti, in quanto il settore in cui lavoriamo \`e in costante evoluzione. Per cui, per colmare almeno una parte della distanza che separa le due realt\`a \`e necessaria una preparazione sui nuovi linguaggi di programmazione usati in ambito lavorativo, in modo da diminuire il tempo di preparazione su linguaggi ormai richiesti ovunque.

\subsection{Valutazione personale}
Per quanto riguarda la mia personale esperienza di stage posso ritenermi pienamente soddisfatto. Tramite l'evento STAGE-IT ho avuto modo di conoscere varie realt\`a aziendali, partendo dalle startup formate da 4-5 persone arrivando ad aziende con all'attivo pi\`u di 80 dipendenti. Ho avuto modo di effettuare diversi colloqui e di visitare in sede diverse aziende. Tutte le realt\`a che ho visto mi hanno incantato. Ho potuto visitare l'ambiente sobrio di startup visitate all'interno di centri di ricerca, assaporare uno stile indipendente e altamente innovativo, ma ho potuto vedere gli ampi e ben arredati spazi di una grande azienda, della seriet\`a con cui lavorano e la voglia che hanno di crescere e migliorare continuamente.\\\\
Fortunatamente, ho potuto osservare quanta speranza \`e ancora riposta in noi giovani, in un settore in cui le persone e soprattutto il valore di una persona viene ancora riconosciuto come una risorsa fondamentale per la crescita di un'azienda.\\
Alla fine, la mia scelta di svolgere lo stage ad Experenti \`e stata ampiamente ripagata dalle numerose soddisfazioni ottenute e dal modo in cui sono stato accolto all'interno dell'azienda, trovando un ambiente accogliente e persone aperte nei miei confronti.\\\\
Un'esperienza sicuramente appagante che mi ha fatto scoprire le mie vere capacit\`a, rendendole disponibili anche agli altri.
Tutto il team di Experenti si \`e dimostrato professionale e disponibile nel supportarmi nella mia formazione, fornendomi spiegazioni dedicate e rispondendo ad ogni mio dubbio. Sono stati molto propensi alle mie proposte, sia contestualmente al progetto da svolgere, sia per l'integrazione di nuovi strumenti di supporto.\\\\
Concludo affermando la mia soddisfazione per il corso di studi scelto, visto non solo in ottica di nozioni apprese ma in una visione di maturazione a livello personale e professionale.
\newpage
\subsection{Screenshot finali}

	\begin{figure}[H]
		\centering
		\includegraphics[width=1\textwidth]{\docsImg s5.jpg}
		\caption{Presentazione iniziale - Ventaglio di legno in apertura}
		\label{fig:Presentazione iniziale - Ventaglio di legno in apertura}
	\end{figure}
	
	\begin{figure}[H]
		\centering
		\includegraphics[width=1\textwidth]{\docsImg s2.jpg}
		\caption{Presentazione iniziale - Avatar}
		\label{fig:Presentazione iniziale - Avatar}
	\end{figure}
		
	\begin{figure}[H]
		\centering
		\includegraphics[width=1\textwidth]{\docsImg s6.jpg}
		\caption{Tutorial - Istruzioni}
		\label{fig:Tutorial - Istruzioni}
	\end{figure}	
	
	\begin{figure}[H]
		\centering
		\includegraphics[width=1\textwidth]{\docsImg s1.jpg}
		\caption{Configuratore - Men\`u inferiore aperto e Avatar in presentazione}
		\label{fig:Configuratore - Menu' inferiore aperto e Avatar in presentazione}
	\end{figure}
	
	\begin{figure}[H]
		\centering
		\includegraphics[width=1\textwidth]{\docsImg s3.jpg}
		\caption{Configuratore - Pannello info aperto}
		\label{fig:Configuratore - Pannello info aperto}
	\end{figure}
		
	\begin{figure}[H]
		\centering
		\includegraphics[width=1\textwidth]{\docsImg s4.jpg}
		\caption{Configuratore - Men\`u dei Credits aperto}
		\label{fig:Configuratore - Menu' dei Credits aperto}
	\end{figure}